\documentclass[]{article}

\usepackage{caption,subcaption,graphicx,float,url,amsmath,amssymb,amsthm,tocloft,cancel,mathrsfs}
\usepackage[toc,nonumberlist]{glossaries}
\usepackage{glossaries-extra,thmtools,gensymb,braket,bm,tensor}
\usepackage[toc,page]{appendix}
\usepackage[T1]{fontenc}
\usepackage[utf8]{inputenc}
\usepackage[toc,page]{appendix}
\newcommand\numberthis{\addtocounter{equation}{1}\tag{\theequation}}
\newcommand{\Lagr}{\mathscr{L}}
\newtheorem{thm}{Theorem}
\newtheorem{defn}[thm]{Definition}
\newtheorem{cor}[thm]{Corollary}
\newtheorem{lemma}[thm]{Lemma}
\graphicspath{{figs/}}
\widowpenalty10000
\clubpenalty10000
\setcounter{tocdepth}{2}

%opening
\title{Theoretical Minimum\\General Relativity\\Exercises}
\author{Simon Crase}

\begin{document}

\maketitle

\begin{abstract}
	These are some exercises arising from the \emph{General Relativity}\cite{susskind2012general} lectures from Leonard Susskind's \emph{Theoretical Minimum} series\cite{susskind2007theoretical}. Section \ref{sec:schwartzchild:metric} contains a derivation of the Schwarzschild metric. Section \ref{sec:gravitational:waves} examines lineratized solutions and gravitational waves.
\end{abstract}

\tableofcontents
\listoffigures
\listoftables
\listoftheorems

	
\section{Schwarzschild Metric}\label{sec:schwartzchild:metric}

The Schwarzschild Metric is presented, without proof, in \cite[Lecture 6]{susskind2012general}. In this section the coordinates are denoted: $(t,r,\theta,\phi)$. The summation convention is observed for all other indices: e.g. $\Gamma\indices{^\alpha_{\gamma\alpha}}$ denoted summation over the dummy index $\alpha$, but $\Gamma\indices{^\theta_{\gamma\theta}}$ is not summed.

\subsection{Ansatz}
We will seek a solution of the form\cite{Adler1965Introduction}:

\begin{align*}
	dt^2 =& e^{\nu(r)} dt^2 -e^{\lambda(r)} dr^2 -r^2 d \Omega^2 \text{, where} \numberthis \label{eq:schwartzschild:ansatz}\\
	d \Omega^2 =&d\theta^2 + cos^2 \theta d\phi^2 
\end{align*}

The  non-zero $g_{\mu\nu}$ are:
\begin{align*}
	g_{tt}=& e^{\nu(r)}\\
	g_{rr}=&-e^{\lambda(r)}\\
	g_{\theta\theta}=&-r^2\\
	g_{\phi\phi}=&-r^2 \cos^2 \theta\\
\end{align*}

The $g^{\mu\nu}$ are the inverse of the $g_{\mu\nu}$, so: 
\begin{align*}
	g^{tt}=& e^{-\nu(r)}\\
	g^{rr}=&-e^{-\lambda(r)}\\
	g^{\theta\theta}=&-\frac{1}{r^2}\\
	g^{\phi\phi}=&-\frac{1}{r^2 \cos^2 \theta}
\end{align*}

The only non-zero derivatives of $g^{\mu\nu}$ are:
\begin{align*}
	\partial_r g_{tt}=& \nu^\prime(r) e^{\nu(r)}\\
	\partial_r g_{rr}=& -\lambda^\prime(r) e^{\lambda(r)}\\
	\partial_r g_{\theta\theta}=& -2 r\\
	\partial_r g_{\phi\phi}=& -2 r \cos^2 \theta\\
	\partial_\theta g_{\phi\phi}=& 2 r^2 \cos \theta \sin \theta
\end{align*}

\subsection{Christoffel Symbols}

We start be enquiring which Christoffel are not identically zero. For each $\rho$ we ask whether there are any $(\mu,\nu)$ for which the terms making up $\Gamma\indices{^\rho_{\mu\nu}}$ are non-zero.

\begin{align*}
	\Gamma\indices{^t_{\mu\nu}} =&\frac{1}{2} \big[\underbrace{ \partial_\mu g_{t\nu}}_\text{$\mu=r,\nu=t$} + \underbrace{\partial_\nu g_{\mu t}}_\text{$\mu=t,\nu=r$} - \underbrace{\partial_t g_{\mu\nu}}_\text{$\equiv0$}\big] e^{-\nu(r)}\\
	\Gamma\indices{^t_{rt}}=&\frac{1}{2} \partial_r g_{tt}  e^{-\nu(r)}\\
	=&\frac{1}{2}\nu^\prime(r) \cancel{e^{\nu(r)}}\cancel{e^{-\nu(r)}}\\
	=&\frac{\nu^\prime(r)}{2}\\
	\Gamma\indices{^t_{tr}}=&\frac{\nu^\prime(r)}{2}\\
	\Gamma\indices{^t_{rt}}=&\Gamma\indices{^t_{tr}}=\frac{\nu^\prime(r)}{2} \numberthis \label{eq:Gamma:ttr}
\end{align*}

\begin{align*}
	\Gamma\indices{^r_{\mu\nu}} =& - \frac{1}{2} \big[\underbrace{ \partial_\mu g_{r\nu}}_\text{$\mu=\nu=r$} + \underbrace{\partial_\nu g_{\mu r}}_\text{ $\mu=\nu=r$} - \underbrace{\partial_r g_{\mu\nu}}_\text{$\mu=\nu$}\big] e^{-\lambda(r)}\\
	\Gamma\indices{^r_{tt}} =&-\frac{1}{2} \big[-\partial_r g_{tt}\big]e^{-\lambda(r)}\\
	=&\frac{1}{2} \big[\nu^\prime(r) e^{\nu(r)}\big] e^{-\lambda(r)}\\
	=& \frac{1}{2} \nu^\prime(r) e^{\nu(r)-\lambda(r)} \numberthis \label{eq:Gamma:rtt}\\
	\Gamma\indices{^r_{rr}} =&-\frac{1}{2}\big[\partial_r g_{rr}\big] e^{-\lambda(r)}\\
	=&\frac{1}{2}\big[\lambda^\prime(r) \cancel{e^{\lambda(r)}}\big] \cancel{e^{-\lambda(r)}}\\
	=&\frac{\lambda^\prime(r)}{2} \numberthis \label{eq:Gamma:rrr}\\
	\Gamma\indices{^r_{\theta\theta}} =&-\frac{1}{2}\big[-\partial_r g_{\theta\theta}\big] e^{-\lambda(r)}\\
	=&\frac{1}{\cancel{2}}\big[-\cancel{2} r \big] e^{-\lambda(r)}\\
	=& -r  e^{-\lambda(r)}\numberthis \label{eq:Gamma:rtHtH}\\
	\Gamma\indices{^r_{\phi\phi}} =&-\frac{1}{2} \big[-\partial_r g_{\phi\phi}\big]e^{-\lambda(r)}\\
	=&\frac{1}{\cancel{2}} \big[-\cancel{2} r \cos^2 \theta\big]e^{-\lambda(r)}\\
	=& -r \cos^2 \theta e^{-\lambda(r)} \numberthis \label{eq:Gamma:rpHpH}
\end{align*}

\begin{align*}
	\Gamma\indices{^\theta_{\mu\nu}} =&- \frac{1}{2} \big[\underbrace{ \partial_\mu g_{\theta\nu}}_\text{$\mu=r,\nu=\theta$} + \underbrace{\partial_\nu g_{\mu \theta}}_\text{$\mu=\theta,\nu=r$} - \underbrace{\partial_\theta g_{\mu\nu}}_\text{$\mu=\nu=\phi$}\big] \frac{1}{r^2} \\
	\Gamma\indices{^\theta_{r\theta}} =& - \frac{1}{2} \big[\partial_r g_{\theta\theta} \big] \frac{1}{r^2}\\
	=& - \frac{1}{2} \big[-2r \big] \frac{1}{r^2}\\
	=& \frac{1}{r} \\
	\Gamma\indices{^\theta_{r\theta}} =& \Gamma\indices{^\theta_{\theta r}} =\frac{1}{r} \numberthis \label{eq:Gamma:tHtHr}\\
	\Gamma\indices{^\theta_{\phi\phi}} =&- \frac{1}{2} \big[-\partial_\theta g_{\phi\phi}\big] \frac{1}{r^2}\\
	=& \frac{1}{\cancel{2}} \big[\cancel{2} \bcancel{r^2} \cos \theta \sin \theta\big] \frac{1}{\bcancel{r^2}}\\
	=& \cos \theta \sin \theta \numberthis \label{eq:Gamma:tHpHpH}		
\end{align*}

\begin{align*}
	\Gamma\indices{^\phi_{\mu\nu}} =&- \frac{1}{2} \big[\underbrace{ \partial_\mu g_{\phi\nu}}_\text{$\mu\in\{\theta,r\},\nu=\phi$} + \underbrace{\partial_\nu g_{\mu \phi}}_\text{$\mu=\phi,\nu\in\{\theta,r\}$} - \underbrace{\partial_\phi g_{\mu\nu}}_\text{$\equiv0$}\big] \frac{1}{r^2 \cos^2 \theta} \\
	\Gamma\indices{^\phi_{\theta\phi}} =&- \frac{1}{2} \big[ \partial_\theta g_{\phi\phi} \big] \frac{1}{r^2 \cos^2 \theta}\\
	=&- \frac{1}{2} \big[ 2 r^2 \cos \theta \sin \theta \big] \frac{1}{r^2 \cos^2 \theta}\\
	=&-\tan \theta \\
	\Gamma\indices{^\phi_{\theta\phi}} =&\Gamma\indices{^\phi_{\phi\theta}} = -\tan \theta \numberthis \label{eq:Gamma:pHpHtH}\\
	\Gamma\indices{^\phi_{r\phi}} =& - \frac{1}{2} \big[\partial_r g_{\phi\phi}\big] \frac{1}{r^2 \cos^2 \theta}	\\
	=& \bcancel{-} \frac{1}{\cancel{2}} \big[\bcancel{-} \cancel{2} r \xcancel{\cos^2 \theta} \big] \frac{1}{r^2 \xcancel{\cos^2 \theta}}	\\
	=& \frac{1}{r}\\
	\Gamma\indices{^\phi_{r\phi}} =& \Gamma\indices{^\phi_{\phi r}} = \frac{1}{r} \numberthis \label{eq:Gamma:pHrpH}
\end{align*}

\subsection{Riemann \& Ricci Tensors}

From \cite[Lecture II]{akhmedov2016lectures}
\begin{align*}
	R\indices{^\mu_{\nu\alpha\beta}} =& \partial_\alpha \Gamma\indices{^\mu_{\nu\beta}} - \partial_\beta \Gamma\indices{^\mu_{\nu\alpha}} + \Gamma\indices{^\mu_{\gamma\alpha}} \Gamma\indices{^\gamma_{\nu\beta}} - \Gamma\indices{^\mu_{\gamma\beta}} \Gamma\indices{^\gamma_{\nu\alpha}}\\
	R_{\mu\nu} =& R\indices{^\alpha_{\mu\alpha\nu}}\\
	=& \partial_\alpha \Gamma\indices{^\alpha_{\mu\nu}} - \partial_\nu \Gamma\indices{^\alpha_{\mu\alpha}} + \Gamma\indices{^\alpha_{\gamma\alpha}} \Gamma\indices{^\gamma_{\mu\nu}} - \Gamma\indices{^\alpha_{\gamma\nu}} \Gamma\indices{^\gamma_{\mu\alpha}}\\
	R =& R_{\mu\nu} g^{\mu\nu}
\end{align*}

We compute the components of $R_{\mu\nu}$.

\subsubsection{$R_{tt}$ and $R_{rr}$}
Since (\ref{eq:schwartzschild:ansatz}) contains two unknown functions, I expect to be able to determine them from two components of $R_{\mu\nu}$.

\begin{align*}
	R_{tt} =& \partial_\alpha \Gamma\indices{^\alpha_{tt}} - \underbrace{\partial_t \Gamma\indices{^\alpha_{t\alpha}}}_\text{$=0$} + \Gamma\indices{^\alpha_{\gamma\alpha}} \Gamma\indices{^\gamma_{tt}} - \Gamma\indices{^\alpha_{\gamma t}} \Gamma\indices{^\gamma_{t\alpha}} \\
	=& \partial_r \Gamma\indices{^r_{tt}}  + \big[\Gamma\indices{^t_{rt}}+\Gamma\indices{^r_{rr}} + \Gamma\indices{^\theta_{r\theta}}+ \Gamma\indices{^\phi_{r\phi}}\big] \Gamma\indices{^r_{tt}} - 2\Gamma\indices{^t_{r t}} \Gamma\indices{^r_{tt}} \\
	=& \partial_r \Gamma\indices{^r_{tt}}  + \big[-\Gamma\indices{^t_{rt}}+\Gamma\indices{^r_{rr}} + \Gamma\indices{^\theta_{r\theta}}+ \Gamma\indices{^\phi_{r\phi}}\big] \Gamma\indices{^r_{tt}} \\
	=& \partial_r \big[\frac{1}{2} \nu^\prime(r) e^{\nu(r)-\lambda(r)}\big]  + \big[-\frac{\nu^\prime(r)}{2}+\frac{\lambda^\prime(r)}{2} + \frac{1}{r} + \frac{1}{r}\big] \frac{1}{2} \nu^\prime(r) e^{\nu(r)-\lambda(r)}  \\
	=& \frac{1}{2} \big[ \nu^{\prime\prime}(r) + \nu^\prime(r) \big(\nu^\prime(r)-\lambda^\prime(r)\big)\big] e^{\nu(r)-\lambda(r)}+ \big[-\frac{\nu^\prime(r)-\lambda^\prime(r)}{2} + \frac{2}{r}\big] \frac{1}{2} \nu^\prime(r) e^{\nu(r)-\lambda(r)} \\
	=& \frac{1}{2} \bigg[\nu^{\prime\prime}(r) + \frac{1}{2}\nu^\prime(r) \big(\nu^\prime(r)-\lambda^\prime(r)\big)+ \frac{2}{r}\nu^\prime(r)\bigg] e^{\nu(r)-\lambda(r)} \numberthis \label{eq:Rtt}
\end{align*}

\begin{align*}
	R_{rr} =& \partial_\alpha \Gamma\indices{^\alpha_{rr}} - \partial_r \Gamma\indices{^\alpha_{r\alpha}} + \Gamma\indices{^\alpha_{\gamma\alpha}} \Gamma\indices{^\gamma_{rr}} - \Gamma\indices{^\alpha_{\gamma r}} \Gamma\indices{^\gamma_{r\alpha}}\\
	=& \underbrace{\partial_t \Gamma\indices{^t_{rr}}}_\text{$=0$} + \cancel{\partial_r \Gamma\indices{^r_{rr}} } +\underbrace{\partial_\theta \Gamma\indices{^\theta_{rr}}}_\text{$=0$} + \underbrace{\partial_\phi \Gamma\indices{^\phi_{rr}}}_\text{$=0$}\\
	&- \partial_r \Gamma\indices{^t_{rt}} - \cancel{\partial_r \Gamma\indices{^r_{rr}}} - \partial_r \Gamma\indices{^\theta_{r\theta}} - \partial_r \Gamma\indices{^\phi_{r\phi}}\\
	&+ \big[\Gamma\indices{^t_{rt}} + \bcancel{\Gamma\indices{^r_{rr}}} + \Gamma\indices{^\theta_{r\theta}}  + \Gamma\indices{^\phi_{r\phi}}\big] \Gamma\indices{^r_{rr}}\\
	&- \Gamma\indices{^t_{t r}} \Gamma\indices{^t_{rt}} - \bcancel{\Gamma\indices{^r_{r r}} \Gamma\indices{^r_{rr}}} - \Gamma\indices{^\theta_{\theta r}} \Gamma\indices{^\theta_{r\theta}} - \Gamma\indices{^\phi_{\phi r}} \Gamma\indices{^\phi_{r\phi}}\\
	=& - \partial_r \frac{\nu^\prime(r)}{2}  - \partial_r \frac{1}{r} - \partial_r \frac{1}{r} + \big[\frac{\nu^\prime(r)}{2}  + \frac{1}{r}  + \frac{1}{r}\big] \frac{\lambda^\prime(r)}{2}- \big(\frac{\nu^\prime(r)}{2}\big)^2 -  \frac{1}{r^2}  - \frac{1}{r^2}\\
	=& -\frac{\nu^{\prime\prime}(r)}{2} + \cancel{\frac{2}{r^2}} + \bigg(\frac{\nu^\prime(r)}{2}  + \frac{2}{r}  \bigg) \frac{\lambda^\prime(r)}{2}- \big(\frac{\nu^\prime(r)}{2}\big)^2 -  \cancel{\frac{2}{r^2}} \\
	=& -\frac{1}{2}\bigg[\nu^{\prime\prime}(r) - \bigg(\frac{\nu^\prime(r)}{2}  + \frac{2}{r}  \bigg) \lambda^\prime(r) + \frac{\big(\nu^\prime(r)\big)^2}{2}\bigg] \numberthis \label{eq:Rrr}
\end{align*}

\subsubsection{Determination of $\nu$ and $\lambda$}

We now have enough information to determine $\nu$ and $\lambda$. From (\ref{eq:Rtt}) and (\ref{eq:Rrr})

\begin{align*}
	\nu^{\prime\prime}(r) + \frac{1}{2}\nu^\prime(r) \big(\nu^\prime(r)-\lambda^\prime(r)\big)+ \frac{2}{r}\nu^\prime(r)=&0 \numberthis \label{eq:Rtt:a}\\
	\nu^{\prime\prime}(r) - \bigg(\frac{\nu^\prime(r)}{2}  + \frac{2}{r}  \bigg) \lambda^\prime(r) + \frac{\big(\nu^\prime(r)\big)^2}{2}=&0\\
\end{align*}
Subtracting these equations:
\begin{align*}
	\cancel{\nu^{\prime\prime}(r)} + \frac{1}{2}\nu^\prime(r) \big(\nu^\prime(r)-\lambda^\prime(r)\big)+ \frac{2}{r}\nu^\prime(r)-\cancel{\nu^{\prime\prime}(r)} + \bigg(\frac{\nu^\prime(r)}{2}  + \frac{2}{r}  \bigg) \lambda^\prime(r) - \frac{\big(\nu^\prime(r)\big)^2}{2}=&0\\
\end{align*}

\begin{align*}
	 \frac{1}{2}\nu^\prime(r) \big(\cancel{\nu^\prime(r)}-\bcancel{\lambda^\prime(r)}\big)+ \frac{2}{r}\nu^\prime(r) + \bigg(\bcancel{\frac{\nu^\prime(r)}{2}}  + \frac{2}{r}  \bigg) \lambda^\prime(r) - \cancel{\frac{\big(\nu^\prime(r)\big)^2}{2}}=&0\\
	 \frac{2}{r}\big(\nu^\prime(r)+\lambda^\prime(r)\big)=&0\\
	 \nu^\prime(r)+\lambda^\prime(r)=&0 \text{, so for some constant $c$}\\
	 e^\lambda(r) =& \frac{c}{e^\nu(r)}  \numberthis \label{eq:lambda:nu}
\end{align*}

Substituting in (\ref{eq:Rtt:a}):
\begin{align*}
	\nu^{\prime\prime}(r) + \nu^\prime(r)^2 + \frac{2}{r}\nu^\prime(r)=&0 \numberthis \label{eq:nu} \text{. Now}\\
	\big[r e^{\nu(r)}\big]^{\prime\prime}=& \big[e^\nu+r \nu^\prime e^\nu\big]^\prime\\
	=& \big[e^\nu \nu^\prime + e^\nu \nu^\prime + r \nu^{\prime\prime} e^\nu + r (\nu^\prime)^2 e^\nu\big]\\
	=& r e^\nu\big[\frac{2}{r} + \nu^{\prime\prime} + (\nu^\prime)^2\big] \text{, so (\ref{eq:nu}) becomes:}\\
	\big[r e^{\nu(r)}\big]^{\prime\prime}=&0, \text{, or}\\
	r e^{\nu(r)} =&ar + b	\text{, for some constants $a$ and $b$}\\
	e^{\nu(r)} =&a + \frac{b}{r}
\end{align*}

Now we know that the metric (\ref{eq:schwartzschild:ansatz}) should be Minkowskian at infinity, which requires $\nu(r)\rightarrow 1$ as $r \rightarrow \infty$, so from:
\begin{align*}
	dt^2 =& \big(a + \frac{b}{r}\big) dt^2 - \frac{c}{a + \frac{b}{r}} dr^2 -r^2 d  \Omega^2 \text{, for some we see that:}\\
	a =& \frac{c}{a}=1 \text{, using (\ref{eq:lambda:nu}), i.e.} \\
	c=1
\end{align*}

We saw in \cite[Lecture 5]{susskind2012general} that we need $g_{00}\approxeq1-\frac{2MG}{r}$ for small $r$, in order to get agreement with Newtonian gravity, hence $b=-2MG$, and (\ref{eq:schwartzschild:ansatz}) becomes:

\begin{align*}
	dt^2 =& \big(1-\frac{2MG}{R}\big) dt^2 - \frac{dr^2}{1-\frac{2MG}{R}}  -r^2 d  \Omega^2 \numberthis \label{eq:schwartzschild:solved}
\end{align*}

\subsubsection{$R_{\theta\theta}$ and $R_{\phi\phi}$}
We need to establish that the remaining diagonal elements are zero using the metric (\ref{eq:schwartzschild:solved}).

\begin{align*}
	R_{\theta\theta} =& \partial_\alpha \Gamma\indices{^\alpha_{\theta\theta}} - \partial_\theta \Gamma\indices{^\alpha_{\theta\alpha}} + \Gamma\indices{^\alpha_{\gamma\alpha}} \Gamma\indices{^\gamma_{\theta\theta}} - \Gamma\indices{^\alpha_{\gamma\theta}} \Gamma\indices{^\gamma_{\theta\alpha}}\\
	=& \partial_r \Gamma\indices{^r_{\theta\theta}} - \partial_\theta \big[\Gamma\indices{^\phi_{\theta\phi}}+\Gamma\indices{^r_{\theta r}}\big] + \bigg(\Gamma\indices{^t_{rt}}+\Gamma\indices{^r_{rr}}+\cancel{\Gamma\indices{^\theta_{r\theta}}}+\Gamma\indices{^\phi_{r\phi}}\bigg) \Gamma\indices{^r_{\theta\theta}}\\ &-\cancel{ \Gamma\indices{^r_{\theta\theta}} \Gamma\indices{^\theta_{\theta r}}}  - \Gamma\indices{^\theta_{r\theta}} \Gamma\indices{^r_{\theta\theta}} -\Gamma\indices{^r_{r\theta}} \Gamma\indices{^r_{\theta r}} - \Gamma\indices{^\phi_{\phi\theta}} \Gamma\indices{^\phi_{\phi\theta}}\\
	=& \partial_r \bigg[-r  e^{-\lambda(r)}\bigg] - \partial_\theta \bigg[-\tan \theta +\frac{1}{r}\bigg]  + \bigg(\underbrace{\frac{\nu^\prime(r)}{2}+\frac{\lambda^\prime(r)}{2}}_\text{$=0$ from (\ref{eq:lambda:nu})}+\cancel{\frac{1}{\bcancel{r}}\bigg)\bigg[ -\bcancel{r}  e^{-\lambda(r)}\bigg]} + \cancel{\frac{r e^{-\lambda(r)}}{r}}	 -\frac{1}{r^2} - \tan^2 \theta 
\end{align*}

 Now, from (\ref{eq:schwartzschild:solved})
\begin{align*}
	e^{\lambda(r)} =& \frac{1}{1 - \frac{2MG}{r}} \text{, whence}\\
	\partial_r \big[r  e^{-\lambda(r)}\big] =& \partial_r \big[r(1 - \frac{2MG}{r})\big]\\
	=& \partial_r [r-2MG]\\
	=& 1 \text{, whence} \numberthis \label{eq:dL}\\
	R_{\theta\theta} =& - 1 + \underbrace{\sec^2 \theta}_\text{$=1+\tan^2 \theta$} + \cancel{\frac{1}{r^2}} -\cancel{\frac{1}{r^2}} - \tan^2 \theta \\
	=& 0 \text{, as expected.}
\end{align*}

Since the calculation of $R_{\phi\phi}$ has proved complex, I have expanded the terms, and used the cancel symbol to mark vanishing Christoffel symbols.

\begin{align*}
	\Gamma\indices{^\alpha_{\gamma\alpha}} \Gamma\indices{^\gamma_{\phi\phi}} =&\Gamma\indices{^\alpha_{t\alpha}} \cancel{\Gamma\indices{^t_{\phi\phi}}}+ \Gamma\indices{^\alpha_{r\alpha}} \Gamma\indices{^r_{\phi\phi}}+\Gamma\indices{^\alpha_{\theta\alpha}} \Gamma\indices{^\theta_{\phi\phi}}+\Gamma\indices{^\alpha_{\phi\alpha}} \cancel{\Gamma\indices{^\phi_{\phi\phi}}}\\
	=&\big[\Gamma\indices{^t_{rt}}+\Gamma\indices{^r_{rr}}+\Gamma\indices{^\theta_{r\theta}}+\Gamma\indices{^\phi_{r\phi}}\big] \Gamma\indices{^r_{\phi\phi}}\\
	+&\big[\cancel{\Gamma\indices{^t_{\theta t}}}+\cancel{\Gamma\indices{^r_{\theta r}}}+\cancel{\Gamma\indices{^\theta_{\theta\theta}}}+\Gamma\indices{^\phi_{\theta\phi}}\big] \Gamma\indices{^\theta_{\phi\phi}}\\
	=& \big[\underbrace{\frac{\nu^\prime}{2} + \frac{\lambda^\prime}{2}}_\text{$=0$}+\frac{1}{r}+\frac{1}{r}\big]\big[-r \cos^2\theta e^{-\lambda}\big] - \tan \theta \cos \theta \sin \theta\\
	=& - 2 \cos^2\theta e^{-\lambda} -\sin^2\theta\numberthis \label{eq:prod:1}
\end{align*}



\begin{align*}
	\Gamma\indices{^\alpha_{\gamma\phi}} \Gamma\indices{^\gamma_{\phi\alpha}}=& \cancel{\Gamma\indices{^t_{t\phi}}} \Gamma\indices{^t_{\phi t}}+\cancel{\Gamma\indices{^t_{r\phi}}} \Gamma\indices{^r_{\phi t}}+\cancel{\Gamma\indices{^t_{\theta\phi}}} \Gamma\indices{^\theta_{\phi t}}+\cancel{\Gamma\indices{^t_{\phi\phi}}} \Gamma\indices{^\phi_{\phi t}}\\
	&+ \cancel{\Gamma\indices{^r_{t\phi}}} \Gamma\indices{^t_{\phi r}} + \cancel{\Gamma\indices{^r_{r\phi}}} \Gamma\indices{^r_{\phi r}}+ \cancel{\Gamma\indices{^r_{\theta\phi}}} \Gamma\indices{^\theta_{\phi r}}+ \Gamma\indices{^r_{\phi\phi}} \Gamma\indices{^\phi_{\phi r}}\\
	&+ \cancel{\Gamma\indices{^\theta_{t\phi}}} \Gamma\indices{^t_{\phi\theta}}+ \cancel{\Gamma\indices{^\theta_{r\phi}}} \Gamma\indices{^r_{\phi\theta}}+ \cancel{\Gamma\indices{^\theta_{\theta\phi}}} \Gamma\indices{^\theta_{\phi\theta}}+ \Gamma\indices{^\theta_{\phi\phi}} \Gamma\indices{^\phi_{\phi\theta}} \\
	&+ \cancel{\Gamma\indices{^\phi_{t\phi}}} \Gamma\indices{^t_{\phi\phi}}+ \Gamma\indices{^\phi_{r\phi}} \Gamma\indices{^r_{\phi\phi}}+ \Gamma\indices{^\phi_{\theta\phi}} \Gamma\indices{^\theta_{\phi\phi}}+ \cancel{\Gamma\indices{^\phi_{\phi\phi}}} \Gamma\indices{^\phi_{\phi\phi}}\\
	=& \Gamma\indices{^r_{\phi\phi}} \Gamma\indices{^\phi_{\phi r}}+  \Gamma\indices{^\theta_{\phi\phi}} \Gamma\indices{^\phi_{\phi\theta}} + \Gamma\indices{^\phi_{r\phi}} \Gamma\indices{^r_{\phi\phi}}+ \Gamma\indices{^\phi_{\theta\phi}} \Gamma\indices{^\theta_{\phi\phi}}\\
	=& 2 \Gamma\indices{^r_{\phi\phi}} \Gamma\indices{^\phi_{\phi r}}+  2 \Gamma\indices{^\theta_{\phi\phi}} \Gamma\indices{^\phi_{\phi\theta}}\\
	=& -2\bcancel{ r} \cos^2 \theta e^{- \lambda(r)} \frac{1}{\bcancel{r}}  - 2 cos\theta \sin\theta \tan\theta\\
	=& -2 \cos^2 \theta e^{- \lambda(r)}   - 2 \bcancel{cos\theta} \sin\theta \frac{\sin\theta}{\bcancel{cos\theta}} \\
	=& -2 \cos^2 \theta e^{- \lambda(r)}   - 2  \sin^2\theta  \numberthis \label{eq:prod:2}
\end{align*}

\begin{align*}
	\partial_\alpha \Gamma\indices{^\alpha_{\phi\phi}} - \partial_\phi \Gamma\indices{^\alpha_{\phi\alpha}} =& \partial_r \big[- \cos^2 \theta r e^{-\lambda(r)}\big]+ \partial_\theta \big[\cos\theta \sin\theta\big]-\underbrace{\partial_\phi \Gamma\indices{^\alpha_{\phi\alpha}}}_\text{= 0}\\
	=& - \cos^2 \theta \underbrace{\partial_r \big[ r e^{-\lambda(r)}\big]}_\text{$=1$}+ \partial_\theta \big[\cos\theta \sin\theta\big]\\
	=& - \bcancel{\cos^2 \theta} -\sin^2\theta +\bcancel{\cos^2 \theta} \numberthis \label{eq:phi:phi:deriv}
\end{align*}

\begin{align*} 
	R_{\phi\phi} =& \partial_\alpha \Gamma\indices{^\alpha_{\phi\phi}} - \partial_\phi \Gamma\indices{^\alpha_{\phi\alpha}} + \Gamma\indices{^\alpha_{\gamma\alpha}} \Gamma\indices{^\gamma_{\phi\phi}} - \Gamma\indices{^\alpha_{\gamma\phi}} \Gamma\indices{^\gamma_{\phi\alpha}}\\
	=&\underbrace{ -\sin^2\theta}_\text{ (\ref{eq:phi:phi:deriv})} - \underbrace{\big[\bcancel{2\cos^2\theta e^{-\lambda}} +\sin^2\theta\big]}_\text{(\ref{eq:prod:1})} + \underbrace{\bcancel{2 \cos^2 \theta e^{- \lambda(r)}}   + 2  \sin^2\theta}_\text{(\ref{eq:prod:2})}\\
	=&0
\end{align*}



\subsubsection{Off diagonal elements}
It is easy to show that the of-diagonal terms vanish. With the exception of $R_{t\theta}$ there is no need to expand the $\Gamma$; we merely use our knowledge of which $\Gamma$ and $\partial_\alpha \Gamma$ are non-zero.

\begin{align*}
	R_{tr} =& \partial_\alpha \Gamma\indices{^\alpha_{tr}} - \partial_r \Gamma\indices{^\alpha_{t\alpha}} + \Gamma\indices{^\alpha_{\gamma\alpha}} \Gamma\indices{^\gamma_{tr}} - \Gamma\indices{^\alpha_{\gamma r}} \Gamma\indices{^\gamma_{t\alpha}}\\
	=&\partial_t \Gamma\indices{^t_{tr}} - \underbrace{\partial_r \Gamma\indices{^\alpha_{t\alpha}}}_\text{$=0$} + \underbrace{\Gamma\indices{^\alpha_{t\alpha}}}_\text{$=0$} \Gamma\indices{^t_{tr}} - \underbrace{\Gamma\indices{^t_{r r}}}_\text{$=0$} \Gamma\indices{^r_{tt}} -\underbrace{ \Gamma\indices{^r_{t r}}}_\text{$=0$} \Gamma\indices{^t_{tr}}\\
	=&0
\end{align*}

\begin{align*}
	R_{t\theta} =& \underbrace{\partial_\alpha \Gamma\indices{^\alpha_{t\theta}} - \partial_\theta \Gamma\indices{^\alpha_{t\alpha}} + \Gamma\indices{^\alpha_{\gamma\alpha}} \Gamma\indices{^\gamma_{t\theta}}}_\text{These terms all vanish} - \Gamma\indices{^\alpha_{\gamma\theta}} \Gamma\indices{^\gamma_{t\alpha}}\\
	=&  - \Gamma\indices{^r_{\theta\theta}} \underbrace{\Gamma\indices{^\theta_{tr}}}_\text{$=0$}  - \Gamma\indices{^\theta_{r\theta}} \underbrace{\Gamma\indices{^r_{t\theta}}}_\text{$=0$} \\
	=&0 
\end{align*}

\begin{align*}
	R_{t\phi} =& \underbrace{\partial_\alpha \Gamma\indices{^\alpha_{t\phi}} - \partial_\phi \Gamma\indices{^\alpha_{t\alpha}} + \Gamma\indices{^\alpha_{\gamma\alpha}} \Gamma\indices{^\gamma_{t\phi}} }_\text{These terms all vanish}- \Gamma\indices{^\alpha_{\gamma\phi}} \Gamma\indices{^\gamma_{t\alpha}}\\
	=& - \Gamma\indices{^r_{\phi\phi}} \underbrace{\Gamma\indices{^\phi_{tr}}}_\text{$=0$} - \Gamma\indices{^\phi_{\theta\phi}} \underbrace{\Gamma\indices{^\theta_{t\phi}}}_\text{$=0$}- \Gamma\indices{^\phi_{r\phi}} \underbrace{\Gamma\indices{^r_{t\phi}}}_\text{$=0$}\\
	&=0
\end{align*}

\begin{align*}
	R_{r\theta} =& \partial_\alpha \Gamma\indices{^\alpha_{r\theta}} - \partial_\theta \Gamma\indices{^\alpha_{r\alpha}} + \Gamma\indices{^\alpha_{\gamma\alpha}} \Gamma\indices{^\gamma_{r\theta}} - \Gamma\indices{^\alpha_{\gamma\theta}} \Gamma\indices{^\gamma_{r\alpha}}\\
	=& \underbrace{\partial_r \Gamma\indices{^r_{r\theta}}}_\text{$=0$} - \underbrace{\partial_\theta \big(\Gamma\indices{^\theta_{r\theta}}+\Gamma\indices{^r_{rr}}\big)}_\text{$=0$} + \Gamma\indices{^\phi_{\theta\phi}} \Gamma\indices{^\theta_{r\theta}} - \Gamma\indices{^r_{\theta\theta}} \underbrace{\Gamma\indices{^\theta_{rr}}}_\text{$=0$} - \Gamma\indices{^\theta_{r\theta}} \underbrace{\Gamma\indices{^r_{r\theta}}}_\text{$=0$}- \Gamma\indices{^\phi_{\phi\theta}} \Gamma\indices{^\phi_{r\phi}}\\
	=&-\frac{1}{r^2} -\cancel{\frac{1}{r}\tan \theta} + \cancel{\frac{1}{r}\tan \theta}\\
	=& 0
\end{align*}

\begin{align*}
	R_{r\phi} =& \partial_\alpha \Gamma\indices{^\alpha_{r\phi}} - \partial_\phi \Gamma\indices{^\alpha_{r\alpha}} + \Gamma\indices{^\alpha_{\gamma\alpha}} \Gamma\indices{^\gamma_{r\phi}} - \Gamma\indices{^\alpha_{\gamma\phi}} \Gamma\indices{^\gamma_{r\alpha}}\\
	=& \underbrace{\partial_\phi \Gamma\indices{^\phi_{r\phi}}}_\text{$=0$} - \underbrace{\partial_\phi \Gamma\indices{^\alpha_{r\alpha}}}_\text{$=0$} + \underbrace{\Gamma\indices{^\alpha_{\phi\alpha}}}_\text{$=0$} \Gamma\indices{^\phi_{r\phi}} - \Gamma\indices{^\phi_{\theta\phi}} \underbrace{\Gamma\indices{^\theta_{r\phi}}}_\text{$=0$} - \Gamma\indices{^\phi_{r\phi}} \underbrace{\Gamma\indices{^r_{r\phi}}}_\text{$=0$}\\
	=&0
\end{align*}

\begin{align*}
	R_{\theta\phi} =& \partial_\alpha \Gamma\indices{^\alpha_{\theta\phi}} - \partial_\phi \Gamma\indices{^\alpha_{\theta\alpha}} + \Gamma\indices{^\alpha_{\gamma\alpha}} \Gamma\indices{^\gamma_{\theta\phi}} - \Gamma\indices{^\alpha_{\gamma\phi}} \Gamma\indices{^\gamma_{\theta\alpha}}\\
	=& \underbrace{\partial_\phi \Gamma\indices{^\phi_{\theta\phi}}}_\text{$=0$} - \underbrace{\partial_\phi \Gamma\indices{^\alpha_{\theta\alpha}}}_\text{$=0$} + \underbrace{\Gamma\indices{^\alpha_{\phi\alpha}}}_\text{$=0$} \Gamma\indices{^\phi_{\theta\phi}} - \underbrace{\Gamma\indices{^\phi_{\phi\phi}}}_\text{$=0$} \Gamma\indices{^\phi_{\theta\phi}}\\
	=&0
\end{align*}

\section{Gravitational waves}\label{sec:gravitational:waves}

Gravitational waves are presented, in \cite[Lecture 10]{susskind2012general}.
In this section the coordinates are denoted: $(t,x,y,x)$. The summation convention is observed for all other indices: e.g. $\Gamma\indices{^\alpha_{\gamma\alpha}}$ denotes summation over the dummy index $\alpha$, but $\Gamma\indices{^x_{\gamma x}}$ is not summed.
\subsection{Linearized Field Equations}

From \cite[Lecture 10]{susskind2012general}:
\begin{align*}
	\Gamma\indices{^{\tau}_{\mu\nu}} =&\frac{1}{2} \big[ \partial_\mu g_{\sigma\nu} + \partial_\nu g_{\mu\sigma} - \partial_\sigma g_{\mu\nu}\big] g^{\sigma\tau}\\
	=& \frac{1}{2} \big[ \partial_\mu h_{\sigma\nu} + \partial_\nu h_{\mu\sigma} - \partial_\sigma h_{\mu\nu}\big] \big[\eta^{\sigma\tau }- O(h)\big]\\
	=& \frac{1}{2} \big[ \partial_\mu h_{\sigma\nu} + \partial_\nu h_{\mu\sigma} - \partial_\sigma h_{\mu\nu}\big] \eta^{\sigma\tau }- O(h^2) \numberthis \label{eq:Gamma:small}
\end{align*}

\subsection{Riemann and Ricci Tensors}
From \cite[Lecture 3]{susskind2012general}:
\begin{align*}
	R\indices{^\alpha_{\beta\gamma\delta}} =& \partial_\gamma\Gamma\indices{^\alpha_{\beta\delta}}-\partial_\delta\Gamma\indices{^\alpha_{\beta\gamma}} + O(\Gamma^2) \text{. Now using $o(\Gamma)=o(h)$ from (\ref{eq:Gamma:small})} \\
	=& \frac{1}{2} \big[ \partial_\gamma\partial_\beta h_{\sigma\delta} + \partial_\gamma\partial_\delta h_{\beta\sigma} - \partial_\gamma\partial_\sigma h_{\beta\delta}\big] \eta^{\sigma\alpha }- \frac{1}{2} \big[ \partial_\delta\partial_\gamma h_{\sigma\beta} + \partial_\delta\partial_\beta h_{\gamma\sigma} - \partial_\delta\partial_\sigma h_{\beta\gamma}\big] \eta^{\sigma\alpha }+ O(h^2)\\
	=& \frac{1}{2} \big[ \partial_\gamma\partial_\beta h_{\sigma\delta} + \partial_\gamma\partial_\delta h_{\beta\sigma} - \partial_\gamma\partial_\sigma h_{\beta\delta} - \partial_\delta\partial_\gamma h_{\sigma\beta} - \partial_\delta\partial_\beta h_{\gamma\sigma} + \partial_\delta\partial_\sigma h_{\beta\gamma}\big] \eta^{\sigma\alpha }+ O(h^2)
\end{align*}
Dropping $O(h^2)$ and contracting
\begin{align*}
	R_{\beta\delta} =& 	R\indices{^\alpha_{\beta\alpha\delta}}\\
	=& \frac{1}{2} \big[ \partial_\alpha\partial_\beta h_{\sigma\delta} + \cancel{\partial_\alpha\partial_\delta h_{\beta\sigma}} - \partial_\alpha\partial_\sigma h_{\beta\delta} - \cancel{\partial_\delta\partial_\alpha h_{\sigma\beta}} - \partial_\delta\partial_\beta h_{\alpha\sigma} + \partial_\delta\partial_\sigma h_{\beta\alpha}\big] \eta^{\sigma\alpha }
	\\
	=& \frac{1}{2} \big[ \partial_\alpha\partial_\beta h_{\sigma\delta} - \partial_\alpha\partial_\sigma h_{\beta\delta} -  \partial_\delta\partial_\beta h_{\alpha\sigma} + \partial_\delta\partial_\sigma h_{\beta\alpha}\big] \eta^{\sigma\alpha } \numberthis \label{eq:linearized}\\
	=& 0 \text{ in empty space}.
\end{align*}
Now $\eta^{00}=-1$, $\eta^{ii}=+1$, and all other values are zero, so Einstein's field equations become:
\begin{align*}
	\sum_i\big[\underbrace{\partial_i\partial_\beta h_{i\delta}}_\text{(a)} - \underbrace{\partial_i\partial_i h_{\beta\delta}}_\text{(b)} -  \underbrace{\partial_\delta\partial_\beta h_{ii}}_\text{(c)} + \underbrace{\partial_\delta\partial_i h_{\beta i}\big]}_\text{(d)}\\
	=\underbrace{\partial_0\partial_\beta h_{0\delta}}_\text{(e)} - \underbrace{\partial_0\partial_0 h_{\beta\delta}}_\text{(f)} -  \underbrace{\partial_\delta\partial_\beta h_{00}}_\text{(g)} + \underbrace{\partial_\delta\partial_0 h_{\beta 0}}_\text{(h)} \numberthis \label{eq:expanded_h}
\end{align*}

\subsection{Ansatz}

We will look for plane wave solutions propagating along the $z$ axis.

\begin{align*}
	h_{\mu\nu}(x,t)=&h^0_{\mu\nu} \sin\big(k(t-z)\big)\\
	\partial_0^2 h_{\mu\nu}=& -k^2 h_{\mu\nu}\\
	\partial_z^2 h_{\mu\nu}=& -k^2 h_{\mu\nu}\\
	\partial_0 \partial_z  h_{\mu\nu}=& k^2 h_{\mu\nu}\\
	\partial_z \partial_0  h_{\mu\nu}=& k^2 h_{\mu\nu}
\end{align*}
All other partial derivatives are zero.

The terms on the left hand side of (\ref{eq:expanded_h}) become:
\begin{align*}
	(a)	\sum_i \partial_i\partial_\beta h_{i\delta} =& \begin{cases}
			\partial_z\partial_\beta h_{z\delta} \text{, $\beta\in \{z,t\}$}\\
			0 \text{, otherwise.}
		\end{cases}\\
	(b)	\sum_i \partial_i\partial_i h_{\beta\delta} =& \partial_z\partial_z h_{\beta\delta}\\
	(c)	\sum_i \partial_\delta \partial_\beta h_{ii} =&\begin{cases}
	\partial_\delta \partial_\beta \sum_i h_{ii} \text{, $\beta\&\delta\in\{z,t\}$}\\
	0 \text{, otherwise}.
	\end{cases}\\
	(d)	\sum_i \partial_\delta\partial_i h_{\beta i} =& \begin{cases}
	\partial_\delta\partial_z h_{\beta z} \text {, $\delta\in\{z,t\}$}\\
	0 \text{, otherwise }
	\end{cases}
\end{align*}

The terms on the right hand side of (\ref{eq:expanded_h}) become:
\begin{align*}
	(e)\;	\partial_0\partial_\beta h_{0\delta}=&0 \text{ unless $\beta\in\{z,t\}$ }\\
	(f)\;	\partial_0\partial_0 h_{\beta\delta}\ne&0\\
	(g)\;	\partial_\delta\partial_\beta h_{00}=&0 \text{ unless $\beta\&\delta\in\{z,t\}$ }\\
	(h)\;	\partial_\delta\partial_0 h_{\beta 0}=&0\text{ unless $\delta\in\{z,t\}$ }&
\end{align*}

\subsection{Detailed Constraints}
Substituting $\beta=0,\delta=0$ in (\ref{eq:expanded_h}):
\begin{align*}
	\sum_i&\big[\underbrace{\partial_i\partial_\beta h_{i\delta}}_\text{(a)} - \underbrace{\partial_i\partial_i h_{\beta\delta}}_\text{(b)} -  \underbrace{\partial_\delta\partial_\beta h_{ii}}_\text{(c)} + \underbrace{\partial_\delta\partial_i h_{\beta i}\big]}_\text{(d)}\\
	&=\underbrace{\partial_0\partial_\beta h_{0\delta}}_\text{(e)}- \underbrace{\partial_0\partial_0 h_{\beta\delta}}_\text{(f)} - \underbrace{\partial_\delta\partial_\beta h_{00}}_\text{(g)} + \underbrace{\partial_\delta\partial_0 h_{\beta 0}}_\text{(h)} \\
	&\underbrace{\partial_z\partial_t h_{zt}}_\text{(a)} - \underbrace{\partial_z\partial_z h_{tt}}_\text{(b)} -  \underbrace{\partial_t\partial_t \sum_i h_{ii}}_\text{(c)} + \underbrace{\partial_t\partial_z h_{tt}}_\text{(d)}\\
	&=\underbrace{\cancel{\partial_t\partial_t h_{tt}}}_\text{(e)}- \underbrace{\cancel{\partial_t\partial_t h_{tt}}}_\text{(f)} - \underbrace{\bcancel{\partial_t\partial_t h_{tt}}}_\text{(g)} + \underbrace{\bcancel{\partial_t\partial_t h_{tt}}}_\text{(h)} \\
	0=&\partial_z\partial_t h_{zt} - \partial_z\partial_z h_{tt} -  \partial_t\partial_t \sum_i h_{ii} + \partial_t\partial_z h_{t z}\\
	=&k^2 \big[h^0_{zt}+h^0_{tt}+\sum_ih^0_{ii}+h^0_{tz}\big] \sin\big(k(t-z)\big)\\
	0=& 2h^0_{zt}+h^0_{tt}+\sum_ih^0_{ii} \numberthis \label{eq:con:00}
	\end{align*}
	
	Substituting $\beta=0,\delta=1$ in (\ref{eq:expanded_h}):
	\begin{align*}
	\sum_i&\big[\underbrace{\partial_i\partial_\beta h_{i\delta}}_\text{(a)} - \underbrace{\partial_i\partial_i h_{\beta\delta}}_\text{(b)} -  \underbrace{\partial_\delta\partial_\beta h_{ii}}_\text{(c)} + \underbrace{\partial_\delta\partial_i h_{\beta i}}_\text{(d)}\big]\\
	&=\underbrace{\partial_0\partial_\beta h_{0\delta}}_\text{(e)} - \underbrace{\partial_0\partial_0 h_{\beta\delta}}_\text{(f)} -  \underbrace{\partial_\delta\partial_\beta h_{00}}_\text{(g)} + \underbrace{\partial_\delta\partial_0 h_{\beta 0}}_\text{(h)}\\
	&\partial_z\partial_t h_{zx}-\partial_z\partial_z h_{tx}=\cancel{\partial_t\partial_t h_{tx}}-\cancel{\partial_t\partial_t h_{tx}}\\
	0=&k^2 \big[h^0_{zx}+h^0_{tx}\big] \sin\big(k(t-z)\big)\\
	0=&h^0_{zx}+h^0_{tx}  \numberthis \label{eq:con:01}
\end{align*}

Substituting $\beta=0,\delta=2$ in (\ref{eq:expanded_h}):
\begin{align*}
	\sum_i&\big[\underbrace{\partial_i\partial_\beta h_{i\delta}}_\text{(a)} - \underbrace{\partial_i\partial_i h_{\beta\delta}}_\text{(b)} -  \underbrace{\partial_\delta\partial_\beta h_{ii}}_\text{(c)} + \underbrace{\partial_\delta\partial_i h_{\beta i}}_\text{(d)}\big]\\
	&=\underbrace{\partial_0\partial_\beta h_{0\delta}}_\text{(e)} - \underbrace{\partial_0\partial_0 h_{\beta\delta}}_\text{(f)} -  \underbrace{\partial_\delta\partial_\beta h_{00}}_\text{(g)} + \underbrace{\partial_\delta\partial_0 h_{\beta 0}}_\text{(h)}\\
	&\underbrace{\partial_z\partial_t h_{zy}}_\text{(a)} - \underbrace{\partial_z\partial_z h_{ty}}_\text{(b)} =\underbrace{\cancel{\partial_t\partial_t h_{ty}}}_\text{(e)} - \underbrace{\cancel{\partial_t\partial_t h_{ty}}}_\text{(f)}\\
	0=&k^2 \big[h^0_{zy}+h^0_{ty}\big] \sin\big(k(t-z)\big)\\
	0=& h^0_{zy}+h^0_{ty} \numberthis \label{eq:con:02}
\end{align*}

Substituting $\beta=0,\delta=3$ in (\ref{eq:expanded_h}):
\begin{align*}
	\sum_i&\big[\underbrace{\partial_i\partial_\beta h_{i\delta}}_\text{(a)} - \underbrace{\partial_i\partial_i h_{\beta\delta}}_\text{(b)} -  \underbrace{\partial_\delta\partial_\beta h_{ii}}_\text{(c)} + \underbrace{\partial_\delta\partial_i h_{\beta i}}_\text{(d)}\big]\\
	&=\underbrace{\partial_0\partial_\beta h_{0\delta}}_\text{(e)} - \underbrace{\partial_0\partial_0 h_{\beta\delta}}_\text{(f)} -  \underbrace{\partial_\delta\partial_\beta h_{00}}_\text{(g)} + \underbrace{\partial_\delta\partial_0 h_{\beta 0}}_\text{(h)}\\
	&\underbrace{\partial_z\partial_t h_{zz}}_\text{(a)} - \underbrace{\xcancel{\partial_z\partial_z h_{tz}}}_\text{(b)} -  \underbrace{\partial_z\partial_t \sum_i h_{ii}}_\text{(c)} + \underbrace{\xcancel{\partial_z\partial_z h_{tz}}}_\text{(d)}\\
	&=\underbrace{\cancel{\partial_t\partial_t h_{tz}}}_\text{(e)} - \underbrace{\cancel{\partial_t\partial_t h_{tz}}}_\text{(f)} -  \underbrace{\bcancel{\partial_z\partial_t h_{tt}}}_\text{(g)} + \underbrace{\bcancel{\partial_z\partial_t h_{tt}}}_\text{(h)}\\
	0=&k^2 \big[h^0_{xx}+h^0_{yy}\big] \sin\big(k(t-z)\big)\\
	0=&h^0_{xx}+h^0_{yy} \numberthis \label{eq:con:03}
\end{align*}

Substituting $\beta=1,\delta=1$ in (\ref{eq:expanded_h}):
\begin{align*}
	\sum_i&\big[\underbrace{\partial_i\partial_\beta h_{i\delta}}_\text{(a)} - \underbrace{\partial_i\partial_i h_{\beta\delta}}_\text{(b)} -  \underbrace{\partial_\delta\partial_\beta h_{ii}}_\text{(c)} + \underbrace{\partial_\delta\partial_i h_{\beta i}}_\text{(d)}\big]\\
	&=\underbrace{\partial_0\partial_\beta h_{0\delta}}_\text{(e)} - \underbrace{\partial_0\partial_0 h_{\beta\delta}}_\text{(f)} -  \underbrace{\partial_\delta\partial_\beta h_{00}}_\text{(g)} + \underbrace{\partial_\delta\partial_0 h_{\beta 0}}_\text{(h)}\\
	&\underbrace{0}_\text{(a)} - \underbrace{\partial_z\partial_z h_{\beta\delta}}_\text{(b)} -  \underbrace{0}_\text{(c)} + \underbrace{0}_\text{(d)}\\
	&=\underbrace{0}_\text{(e)} - \underbrace{\partial_0\partial_0 h_{\beta\delta}}_\text{(f)} -  \underbrace{0}_\text{(g)} + \underbrace{0}_\text{(h)}\\
	0=&\partial_z\partial_z h_{xx}-\partial_0\partial_0 h_{xx}\\
	0=&k^2 \big[ h^0_{xx} - h^0_{xx}\big] \sin\big(k(t-z)\big) \text{ an empty constraint }
\end{align*}

Substituting $\beta=1,\delta=2$ in (\ref{eq:expanded_h}):
\begin{align*}
	\sum_i&\big[\underbrace{\partial_i\partial_\beta h_{i\delta}}_\text{(a)} - \underbrace{\partial_i\partial_i h_{\beta\delta}}_\text{(b)} -  \underbrace{\partial_\delta\partial_\beta h_{ii}}_\text{(c)} + \underbrace{\partial_\delta\partial_i h_{\beta i}}_\text{(d)}\big]\\
	&=\underbrace{\partial_0\partial_\beta h_{0\delta}}_\text{(e)} - \underbrace{\partial_0\partial_0 h_{\beta\delta}}_\text{(f)} -  \underbrace{\partial_\delta\partial_\beta h_{00}}_\text{(g)} + \underbrace{\partial_\delta\partial_0 h_{\beta 0}}_\text{(h)}\\
	&\underbrace{0}_\text{(a)} - \underbrace{\partial_z\partial_z h_{xy}}_\text{(b)} -  \underbrace{0}_\text{(c)} + \underbrace{0}_\text{(d)}\\
	&=\underbrace{0}_\text{(e)} - \underbrace{\partial_0\partial_0 h_{xy}}_\text{(f)} -  \underbrace{0}_\text{(g)} + \underbrace{0}_\text{(h)}\\
	0=& \partial_z\partial_z h_{\beta\delta} -\partial_0\partial_0 h_{\beta\delta}\\
	0=&k^2 \big[h^0_{xy}-h^0_{xy}\big] \sin\big(k(t-z)\big) \text{ an empty constraint }
\end{align*}

Substituting $\beta=1,\delta=3$ in (\ref{eq:expanded_h}):
\begin{align*}
	\sum_i&\big[\underbrace{\partial_i\partial_\beta h_{i\delta}}_\text{(a)} - \underbrace{\partial_i\partial_i h_{\beta\delta}}_\text{(b)} -  \underbrace{\partial_\delta\partial_\beta h_{ii}}_\text{(c)} + \underbrace{\partial_\delta\partial_i h_{\beta i}}_\text{(d)}\big]\\
	&=\underbrace{\partial_0\partial_\beta h_{0\delta}}_\text{(e)} - \underbrace{\partial_0\partial_0 h_{\beta\delta}}_\text{(f)} -  \underbrace{\partial_\delta\partial_\beta h_{00}}_\text{(g)} + \underbrace{\partial_\delta\partial_0 h_{\beta 0}}_\text{(h)}\\
	&\underbrace{0}_\text{(a)} - \underbrace{\cancel{\partial_z\partial_z h_{xz}}}_\text{(b)} -  \underbrace{0}_\text{(c)} + \underbrace{\cancel{\partial_z\partial_z h_{x z}}}_\text{(d)}\\
	&=\underbrace{0}_\text{(e)} - \underbrace{\partial_t\partial_t h_{xz}}_\text{(f)} -  \underbrace{0}_\text{(g)} + \underbrace{\partial_z\partial_t h_{xt}}_\text{(h)}\\
	0=&\partial_t\partial_t h_{xz}-\partial_z\partial_t h_{xt}\\
	=&k^2 \big[h^0_{xz}+h^0_{xt}\big] \sin\big(k(t-z)\big)\\
	0=&h^0_{xz}+h^0_{xt} \text{, which is the same as (\ref{eq:con:01})}
\end{align*}

Substituting $\beta=2,\delta=2$ in (\ref{eq:expanded_h}):
\begin{align*}
	\sum_i&\big[\underbrace{\partial_i\partial_\beta h_{i\delta}}_\text{(a)} - \underbrace{\partial_i\partial_i h_{\beta\delta}}_\text{(b)} -  \underbrace{\partial_\delta\partial_\beta h_{ii}}_\text{(c)} + \underbrace{\partial_\delta\partial_i h_{\beta i}}_\text{(d)}\big]\\
	&=\underbrace{\partial_0\partial_\beta h_{0\delta}}_\text{(e)} - \underbrace{\partial_0\partial_0 h_{\beta\delta}}_\text{(f)} -  \underbrace{\partial_\delta\partial_\beta h_{00}}_\text{(g)} + \underbrace{\partial_\delta\partial_0 h_{\beta 0}}_\text{(h)}\\
	&\underbrace{0}_\text{(a)} - \underbrace{\partial_z\partial_z h_{yy}}_\text{(b)} -  \underbrace{0}_\text{(c)} + \underbrace{0}_\text{(d)}\\
	&=\underbrace{0}_\text{(e)} - \underbrace{\partial_t\partial_t h_{yy}}_\text{(f)} -  \underbrace{0}_\text{(g)} + \underbrace{0}_\text{(h)}\\
	0=&\partial_z\partial_z h_{yy}-\partial_t\partial_t h_{yy}\\
	=&k^2 \big[h^0_{yy}-h^0_{yy}\big] \sin\big(k(t-z)\big) \\
	0=&h^0_{yy}-h^0_{yy} \text{ an empty constraint}
\end{align*}

Substituting $\beta=2,\delta=3$ in (\ref{eq:expanded_h}):
\begin{align*}
	\sum_i&\big[\underbrace{\partial_i\partial_\beta h_{i\delta}}_\text{(a)} - \underbrace{\partial_i\partial_i h_{\beta\delta}}_\text{(b)} -  \underbrace{\partial_\delta\partial_\beta h_{ii}}_\text{(c)} + \underbrace{\partial_\delta\partial_i h_{\beta i}}_\text{(d)}\big]\\
	&=\underbrace{\partial_0\partial_\beta h_{0\delta}}_\text{(e)} - \underbrace{\partial_0\partial_0 h_{\beta\delta}}_\text{(f)} -  \underbrace{\partial_\delta\partial_\beta h_{00}}_\text{(g)} + \underbrace{\partial_\delta\partial_0 h_{\beta 0}}_\text{(h)}\\
	&\underbrace{0}_\text{(a)} - \underbrace{\cancel{\partial_z\partial_z h_{yz}}}_\text{(b)} -  \underbrace{0}_\text{(c)} + \underbrace{\cancel{\partial_z\partial_z h_{yz}}}_\text{(d)}\\
	&=\underbrace{0}_\text{(e)} - \underbrace{\partial_t\partial_t h_{yz}}_\text{(f)} -  \underbrace{0}_\text{(g)} + \underbrace{\partial_z\partial_t h_{yt}}_\text{(h)}\\
	0=&\partial_t\partial_t h_{yz}-\partial_z\partial_t h_{yt}\\
	=&k^2 \big[ h_{yz}+h_{yt}\big] \sin\big(k(t-z)\big) \\
	0=&  h_{yz}+h_{yt} \text{, which is the same as (\ref{eq:con:02})}
\end{align*}

Substituting $\beta=3,\delta=3$ in (\ref{eq:expanded_h}):
\begin{align*}
	\sum_i&\big[\underbrace{\partial_i\partial_\beta h_{i\delta}}_\text{(a)} - \underbrace{\partial_i\partial_i h_{\beta\delta}}_\text{(b)} -  \underbrace{\partial_\delta\partial_\beta h_{ii}}_\text{(c)} + \underbrace{\partial_\delta\partial_i h_{\beta i}}_\text{(d)}\big]\\
	&=\underbrace{\partial_0\partial_\beta h_{0\delta}}_\text{(e)} - \underbrace{\partial_0\partial_0 h_{\beta\delta}}_\text{(f)} -  \underbrace{\partial_\delta\partial_\beta h_{00}}_\text{(g)} + \underbrace{\partial_\delta\partial_0 h_{\beta 0}}_\text{(h)}\\
	&\underbrace{\cancel{\partial_z\partial_z h_{zz}}}_\text{(a)} - \underbrace{\cancel{\partial_z\partial_z h_{zz}}}_\text{(b)} -  \underbrace{\partial_z\partial_z \sum_i h_{ii}}_\text{(c)} + \underbrace{\partial_z\partial_z h_{zz}}_\text{(d)}\\
	&=\underbrace{\partial_t\partial_z h_{tz}}_\text{(e)} - \underbrace{\partial_t\partial_t h_{zz}}_\text{(f)} -  \underbrace{\partial_z\partial_z h_{tt}}_\text{(g)} + \underbrace{\partial_z\partial_t h_{zt}}_\text{(h)}\\
	0=& \partial_z\partial_z\big(h_{xx}+h_{yy}\big) + \partial_t\partial_z h_{tz} - \partial_t\partial_t h_{zz} - \partial_z\partial_z h_{tt} +\partial_z\partial_t h_{zt}\\
	=&k^2 \big[-h^0_{xx} - h^0_{yy} + h^0_{tz} +  h^0_{zz} + h^0_{tt} + h^0_{zt}\big] \sin\big(k(t-z)\big)l_\delta\partial_\beta h_{00} - \partial_\delta\partial_0 h_{\beta 0} \\
	0=&\underbrace{-h^0_{xx} - h^0_{yy}}_\text{$=0$ from (\ref{eq:con:03})} + h^0_{tz} +  h^0_{zz} + h^0_{tt} + h^0_{zt}\\
	=&  +  h^0_{zz} + h^0_{tt} + 2h^0_{zt} \numberthis \label{eq:con:33}
\end{align*}	

\subsection{Summary of Constraints}

Table \ref{table:constraints} shows there are only 5 constraints on the 10 parameters $h^0_{\mu\nu}$, so there are indeed plane wave solutions.

\begin{table}[H]
	\begin{center}
		\caption{Constraints on the parameters $h^0_{\mu\nu}$}\label{table:constraints}
		\begin{tabular}{|c|l|} \hline
			\#&Constraint\\ \hline
			(\ref{eq:con:00})&$2h^0_{zt}+h^0_{tt}+\underbrace{h^0_{xx}+h^0_{yy}}_\text{$=0$ from (\ref{eq:con:03})}+h^0_{zz} =0$\\ \hline
			(\ref{eq:con:01})&$h^0_{zx}+h^0_{tx}=0$\\ \hline
			(\ref{eq:con:02})&$h^0_{zy}+h^0_{ty}=0$\\ \hline
			(\ref{eq:con:03})&$h^0_{xx}+h^0_{yy}=0$\\ \hline
			(\ref{eq:con:33})&$h^0_{zz} + h^0_{tt} + 2h^0_{zt}=0$\\ \hline
		\end{tabular}
	\end{center}
\end{table}
	
	
	

\bibliographystyle{unsrt}
\addcontentsline{toc}{section}{Bibliography}
\raggedright
\bibliography{tm}
\end{document}
