% !TeX program = lualatex
\documentclass[]{article}

\usepackage{caption,subcaption,graphicx,float,url,amsmath,amssymb,amsthm,tocloft,cancel,thmtools,gensymb,braket,tikz-feynman,mathtools,color, colortbl}
\usepackage[toc,nonumberlist]{glossaries}
\usepackage{glossaries-extra}
\usepackage[toc,page]{appendix}

\newcommand\numberthis{\addtocounter{equation}{1}\tag{\theequation}}

\newtheorem{thm}{Theorem}
\newtheorem{defn}[thm]{Definition}
\newtheorem{cor}[thm]{Corollary}
\newtheorem{lemma}[thm]{Lemma}
\graphicspath{{figs/}}
\widowpenalty10000
\clubpenalty10000
\setcounter{tocdepth}{2}
\tikzfeynmanset{compat=1.1.0}
\definecolor{Gray}{gray}{0.5}
\DeclareMathOperator{\Tr}{Tr \;}
%opening
\title{QFT}
\author{Simon Crase (compiler)\\simon@greenweaves.nz}



\begin{document}

\maketitle

\begin{abstract}

This document contains derivations of equations from \cite{zee2010quantum}.

\end{abstract}

\section{Motivation and Foundation}
\subsection{Path Integral Formulation}

\begin{align*}
	\int_{-\infty}^{\infty} dx e^{-\frac{1}{2}ax^2} =& \frac{1}{\sqrt{a}}\int_{-\infty}^{\infty} \underbrace{dy e^{-\frac{1}{2}y^2}}_\text{ where $y=\sqrt{a}x$} \\
	=& \bigg(\frac{2\pi}{a}\bigg)^\frac{1}{2} \numberthis \label{eq:integral:gaussian}
\end{align*}
\begin{thm}
	\begin{align*}
		\frac{\int_{-\infty}^\infty dx e^{-\frac{1}{2}ax^2}x^{2n}}{\int_{-\infty}^\infty dx e^{-\frac{1}{2}ax^2}} =&\frac{\prod_{k=1}^{n}\big(2k-1\big)}{a^n}
	\end{align*}
\end{thm}

\begin{proof}
	The proof by mathematical induction starts by defining the proposition $P(n)$:
	\begin{align*}
		P(n) \equiv& \bigg[\frac{\int_{-\infty}^\infty dx e^{-\frac{1}{2}ax^2}x^{2n}}{\int_{-\infty}^\infty dx e^{-\frac{1}{2}ax^2}}=\frac{\prod_{k=1}^{n}\big(2k-1\big)}{a^n}\bigg]
	\end{align*}
	\begin{lemma}\label{thm:P_0}
		P(0) is true.
	\end{lemma}
	\begin{proof}
		$P(0)$ reduces to $\frac{\int_{-\infty}^\infty dx e^{-\frac{1}{2}ax^2}}{\int_{-\infty}^\infty dx e^{-\frac{1}{2}ax^2}}=1$
	\end{proof}
    \begin{lemma}\label{lemma:uniform:convergence}
    	\begin{align*}
    		\forall \alpha >0 \text{, }\frac{d}{d\alpha}\int_{-\infty}^\infty dx e^{-\frac{1}{2}\alpha x^2}x^{2n} =&\int_{-\infty}^\infty dx \frac{\partial}{\partial \alpha} \big[e^{-\frac{1}{2}\alpha x^2}x^{2n}\big]
    	\end{align*}
    \end{lemma}
	\begin{proof}
		\begin{align*}
			\forall a>0, \forall \alpha_0 \in (0,\alpha)\; \int_{a}^\infty dx e^{-\frac{1}{2}\alpha x^2}x^{2n} =&\int_{a}^\infty dx e^{-\frac{1}{2}\alpha_0 x^2} e^{-\frac{1}{2}(\alpha-\alpha_0) x^2}x^{2n} \\
			\text{Now define}\; M(x) =& e^{-\frac{1}{2}\alpha_0 x^2}
		\end{align*}
		We can choose $a$ large enough that $e^{-\frac{1}{2}(\alpha-\alpha_0) x^2}x^{2n}$ is monotone decreasing for $x>a$. Clearly
		\begin{align*}
			e^{-\frac{1}{2}\alpha_0 x^2} e^{-\frac{1}{2}(\alpha-\alpha_0) x^2}x^{2n} \in C\\
				e^{-\frac{1}{2}\alpha_0 x^2} \in C\\
			e^{-\frac{1}{2}\alpha_0 x^2} e^{-\frac{1}{2}(\alpha-\alpha_0) x^2}x^{2n}<&M(x)\; \forall x>a \text{ and}\\
			\int_{a}^\infty dx M(x)<&\infty
		\end{align*}
		 for $\alpha$ in closed interval [A,B] that includes $\alpha_0$; the Weierstrass M-Test \cite[Chapter 10, 6.1]{widder1961advanced} shows that the integral converges uniformly for $\alpha\in[A,B]$. Hence we can differentiate under the integral sign \cite[Chapter 10, 8.3]{widder1961advanced}.
		Now
		\begin{align*}
			\int_{-\infty}^\infty dx e^{-\frac{1}{2}\alpha x^2}x^{2n} =& 2 \int_{a}^\infty dx e^{-\frac{1}{2}\alpha x^2}x^{2n} + \int_{-a}^a dx e^{-\frac{1}{2}\alpha x^2}x^{2n}
		\end{align*}
	Noting that $e^{-\frac{1}{2}\alpha x^2}x^{2n} \in C^1$ for x in [-a,a], we can also differentiate the second integral.
	\end{proof}

	\begin{lemma}\label{thm:P_1}
		$P(0) \implies P(1)$
	\end{lemma}
	\begin{proof}
		Since (\ref{eq:integral:gaussian}) converges uniformly, Lemma \ref{lemma:uniform:convergence} allows us to differentiate under the integral sign\cite{widder1961advanced}.
		\begin{align*}
			-2\frac{d}{da}\int_{-\infty}^{\infty} dx e^{-\frac{1}{2}ax^2} =& -2\frac{d}{da} \bigg(\frac{2\pi}{a}\bigg)^\frac{1}{2}\\
			\int_{-\infty}^{\infty} dx e^{-\frac{1}{2}ax^2} x^2 =& \bigg(\frac{2\pi}{a^3}\bigg)^\frac{1}{2}\\
			=& \bigg(\frac{2\pi}{a}\bigg)^\frac{1}{2} \frac{1}{a}\\
			=& \int_{-\infty}^{\infty} dx e^{-\frac{1}{2}ax^2} \frac{1}{a}
		\end{align*}
	\end{proof}
	\begin{lemma}\label{thm:P_n}
		$P(n) \land P(1)\implies P(n+1)$
	\end{lemma}
	\begin{proof}
		\begin{align*}
		P(n) \implies &\\
		\int_{-\infty}^\infty dx e^{-\frac{1}{2}ax^2}x^{2n}=&\prod_{k=1}^{n}\big(2k-1\big)	\int_{-\infty}^\infty dx e^{-\frac{1}{2}ax^2} \frac{1}{a^n}\\
		-2\frac{d}{da}\int_{-\infty}^\infty dx e^{-\frac{1}{2}ax^2}x^{2n}=&-2 \prod_{k=1}^{n}\big(2k-1\big)\frac{d}{da} \big[	\int_{-\infty}^\infty dx e^{-\frac{1}{2}ax^2} \frac{1}{a^n}\big]\\
		\int_{-\infty}^\infty dx e^{-\frac{1}{2}ax^2}x^{2(n+1)}=&-2 \prod_{k=1}^{n}\big(2k-1\big)\bigg[\int_{-\infty}^{\infty} dx e^{-\frac{1}{2}ax^2} (-\frac{1}{2}x^2)\frac{1}{a^n}-\int_{-\infty}^{\infty} dx e^{-\frac{1}{2}ax^2}\frac{n}{a^{n+1}}\bigg] \\
		=&\prod_{k=1}^{n}\big(2k-1\big)\bigg[\underbrace{\int_{-\infty}^{\infty} dx e^{-\frac{1}{2}ax^2}x^2}_\text{Now we apply $P(1)$} \frac{1}{a^n}+2\int_{-\infty}^{\infty} dx e^{-\frac{1}{2}ax^2}\frac{n}{a^{n+1}}\bigg]\\
		=&\prod_{k=1}^{n}\big(2k-1\big)\bigg[ \int_{-\infty}^{\infty} dx e^{-\frac{1}{2}ax^2}\frac{1}{a} \frac{1}{a^n} + \int_{-\infty}^{\infty} dx e^{-\frac{1}{2}ax^2} \frac{2n}{a^{n+1}}\bigg]\\
		=&\prod_{k=1}^{n}\big(2k-1\big) \frac{2n+1}{a^{n+1}}\int_{-\infty}^{\infty} dx e^{-\frac{1}{2}ax^2}\\
		=&\prod_{k=1}^{n+1}\big(2k-1\big) \frac{1}{a^{n+1}}\int_{-\infty}^{\infty} dx e^{-\frac{1}{2}ax^2}\\
		\equiv& P(n+1)
		\end{align*}
	\end{proof}
	Summarizing:
	\begin{align*}
		\text{Lemma }\ref{thm:P_0} \implies& P(0)\\
		P(0) \land \text{Lemma } \ref{thm:P_1} \implies& P(1)\\
		P(0) \land P(1)\land P(n) \land \text{Lemma } \ref{thm:P_n} \implies& P(n+1)
	\end{align*}
\end{proof}

\begin{thm}
	If $A$ is a symmetric matrix
	\begin{align*}
		\int_{-\infty}^{\infty}dx_1 dx_2...dx_N e^{-\frac{1}{2}\vec{x}A\vec{x} + \vec{J} \cdot \vec{x}} =& \bigg(\frac{(2\pi)^N}{\vert A \vert}\bigg)^\frac{1}{2} e^{\frac{1}{2}\vec{J}\cdot A^{-1} \vec{J}} \numberthis \label {eq:Zee_I_2_22}
	\end{align*}
\end{thm}
\begin{proof}
	Since $A$ is symmetric, there exists an orthogonal matrix $O$ such that $A=O^T D O$, where $D$ is a diagonal matrix--\cite{bellman1970introduction}. Define
	\begin{align*}
		y_i =& \sum_j O_{ij}x_j\\
		\int_{-\infty}^{\infty}dx_1 dx_2...dx_N e^{-\frac{1}{2}\vec{x}A\vec{x} + \vec{J} \cdot \vec{x}}=&\bigg(\frac{1}{\vert O \vert}\bigg)^N \int_{-\infty}^{\infty}dy_1 dy_2...dy_Ne^{-\frac{1}{2}\vec{x}O^T D O\vec{x} + \vec{J} \cdot \vec{x}}\\
		=& \int_{-\infty}^{\infty}dy_1 dy_2...dy_N e^{-\frac{1}{2}(\vec{x}O^T) D (O\vec{x}) + \vec{J} \cdot \vec{x}}\\
		=& \int_{-\infty}^{\infty}dy_1 dy_2...dy_N e^{- \frac{1}{2} \vec{y}D \vec{y} + (O\vec{J})\cdot \vec{y}}\\
		=& \prod_{i=1}^N \int_{-\infty}^{\infty}dy_i e^{- \frac{1}{2} D_{ii}y_i^2 + (OJ)_i y_i}\\
		=& \prod_{i=1}^N \bigg[\bigg(\frac{2\pi}{D_{ii}}\bigg)^\frac{1}{2} e^{\big(\frac{[(OJ)_i]^2}{2D_{ii}}\big)}\bigg] \numberthis \label {eq:Zee_I_2_22_a}
	\end{align*}
	But
	\begin{align*}
		\prod_{i=1}^N \bigg(\frac{2\pi}{D_{ii}}\bigg) =& \frac{({2\pi})^N}{\vert D \vert}\\
		=& \frac{({2\pi})^N}{\vert A \vert} \text{, since $O$ is orthogonal} \numberthis \label {eq:Zee_I_2_22_b}
	\end{align*}
	And
	\begin{align*}
		\prod_{i=1}^N e^{\big(\frac{[(OJ)_i]^2}{2D_{ii}}\big)} =& e^{\sum_{i=1}^N {\big(\frac{[(OJ)_i]^2}{2D_{ii}}\big)}}\\
		\sum_{i=1}^N {\big(\frac{[(OJ)_i]^2}{2D_{ii}}\big)} =& \frac{1}{2} \sum_{i=1}^N (OJ)_i (D^{-1})_{ii} (OJ)_i\\
		 =& \frac{1}{2} \sum_{i,j,k,l=1}^N O_{ik} J_k (D^{-1})_{ij} O_{jl} J_l\\
		 =& \frac{1}{2} \sum_{i,j,k,l=1}^N J_k O^T_{ki}  (D^{-1})_{ij} O_{jl} J_l \\
		 =& \frac{1}{2} \sum_{k,l=1}^N J_k \bigg(\sum_{k,l=1}^N O^T_{ki}  (D^{-1})_{ij} O_{jl}\bigg) J_l\\
		  \\
		 =& \frac{1}{2} \sum_{k,l=1}^N J_k \big(A^{-1}\big)_{kl} J_l \\
		 =& \frac{1}{2} J A^{-1} J \numberthis \label {eq:Zee_I_2_22_c}
	\end{align*}
	Substituting (\ref{eq:Zee_I_2_22_b}) and (\ref{eq:Zee_I_2_22_c}) in (\ref{eq:Zee_I_2_22_a}), we obtain (\ref{eq:Zee_I_2_22}).
\end{proof}

\subsection{From Field to Particle to Force}
\begin{align*}
	W(J) =& - \frac{1}{2} \int \int d^4x d^4y J(x) D(x-y) J(y)\\
	D(x-y) =& \int \frac{d^4k}{(2\pi)^4} \frac{e^{ik(x-y)}}{k^2-m^2+i\epsilon}\\
	J(x) =& J_1(x) + J_2(x) \text{, where}\\
	J_{a}(x) =&\delta^{(3)}(\vec{x}-\vec{x_a}) 
\end{align*}
Considering only the cross terms
\begin{align*}
	W(J) =& - \int \frac{1}{(2\pi)^4} \cancel{2} \cancel{\frac{1}{2}} \int \int d^4x d^4y d^4k \delta^{(3)}(\vec{x}-\vec{x_1})   \frac{e^{ik(x-y)}}{k^2-m^2+i\epsilon} \delta^{(3)}(\vec{y}-\vec{x_2}))\\
	=& - \frac{1}{(2\pi)^4} \int \int \int \int \int \int dx^0 d\vec{x} dy^0 d\vec{y} dk^0 d\vec{k}  \delta^{(3)}(\vec{x}-\vec{x_1})   \frac{e^{ik^0(x^0-y^0)-i\vec{k} \cdot (\vec{x}-\vec{y})}}{k^2-m^2+i\epsilon} \delta^{(3)}(\vec{y}-\vec{x_2}))\\
	=& - \frac{1}{(2\pi)^4} \int \int \int \int \int dx^0 d\vec{x} dy^0 dk^0 d\vec{k}  \delta^{(3)}(\vec{x}-\vec{x_1})   \frac{e^{ik^0(x^0-y^0)-i\vec{k} \cdot (\vec{x}-\vec{x_2})}}{k^2-m^2+i\epsilon} \\
	=& - \frac{1}{(2\pi)^4} \int \int \int \int dx^0  dy^0 dk^0 d\vec{k} \;    \frac{e^{ik^0(x^0-y^0)-i\vec{k} \cdot (\vec{x_1}-\vec{x_2})}}{k^2-m^2+i\epsilon}\\
	=& - \frac{1}{(2\pi)^3} \int \int dx^0  dy^0 \underbrace{\int \frac{dk^0}{2\pi} e^{ik^0(x^0-y^0)}}_\text{$\delta(x^0-y^0)$} \int d\vec{k} \; \frac{e^{i\vec{k} \cdot (\vec{x}-\vec{y})}}{k^2-m^2+i\epsilon}\\
	=& - \frac{1}{(2\pi)^3} \int  dx^0  \int d\vec{k} \; \frac{e^{i\vec{k} \cdot (\vec{x}-\vec{y})}}{k^2-m^2+i\epsilon}
\end{align*}
\url{https://youtu.be/aypGTsLN0ck?t=3294}


\bibliographystyle{unsrt}
\addcontentsline{toc}{section}{Bibliography}
\raggedright
\bibliography{tm}

\end{document}
