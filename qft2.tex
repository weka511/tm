% !TeX program = lualatex
\documentclass[]{article}

\usepackage{caption,subcaption,graphicx,float,url,amsmath,amssymb,amsthm,tocloft,cancel,thmtools,gensymb,braket,tikz-feynman,mathtools,color, colortbl}
\usepackage[toc,nonumberlist]{glossaries}
\usepackage{glossaries-extra}
\usepackage[toc,page]{appendix}

\newcommand\numberthis{\addtocounter{equation}{1}\tag{\theequation}}

\newtheorem{thm}{Theorem}
\newtheorem{defn}[thm]{Definition}
\newtheorem{cor}[thm]{Corollary}
\newtheorem{lemma}[thm]{Lemma}
\graphicspath{{figs/}}
\widowpenalty10000
\clubpenalty10000
\setcounter{tocdepth}{2}
\tikzfeynmanset{compat=1.1.0}
\definecolor{Gray}{gray}{0.5}
\DeclareMathOperator{\Tr}{Tr \;}
\newcommand{\Lagr}{\mathcal{L}}
\renewcommand{\thesection}{2.\arabic{section}}
\setlength{\cftsubsecindent}{0em}
\setlength{\cftsecnumwidth}{3em}
\setlength{\cftsubsecnumwidth}{3em}
%opening
\title{Quantum Field Theory\\
Part II: Dirac and the Spinor}
\author{Simon Crase (compiler)\\simon@greenweaves.nz}



\begin{document}

\maketitle

\begin{abstract}

This document contains derivations of equations from \cite[Part II: Dirac and the Spinor]{zee2010quantum}.

\end{abstract}

\tableofcontents
\section{Rotations}
In \cite[Section 3]{westra2008SU2} a $2\times 2$ unimodular unitary matrix has the form:
\begin{align*}
	U =& \begin{bmatrix}
		x&-y\\
		\bar{y}&\bar{x}
	\end{bmatrix} \text{. Substituting}\\
	x =& \cos{\theta} e^{i\phi}, \; y = \sin{\theta} e^{i\psi} \numberthis \label{eq:xy} \text{, gives} \\
	U=& \begin{bmatrix}
		\cos{\theta} e^{i\phi}&\sin{\theta}e^{i\psi}\\
		-sin{\theta}e^{-i\psi}&\cos{\theta} e^{-i\phi}
	\end{bmatrix}\\
	U^T =& \begin{bmatrix}
		\cos{\theta} e^{-i\phi}&-\sin{\theta}e^{i\psi}\\
		sin{\theta}e^{-i\psi}&\cos{\theta} e^{i\phi}
	\end{bmatrix}
\end{align*}
We follow \cite[Section 4]{westra2008SU2} and transform the Pauli Matrices.
\begin{align*}
	U \sigma^1 U^T  =&  \begin{bmatrix}
		\cos{\theta} e^{i\phi}&\sin{\theta}e^{i\psi}\\
		-\sin{\theta}e^{-i\psi}&\cos{\theta} e^{-i\phi}
	\end{bmatrix} \begin{bmatrix}
		0&1\\
		1&0
	\end{bmatrix} \begin{bmatrix}
		\cos{\theta} e^{-i\phi}&-\sin{\theta}e^{i\psi}\\
		\sin{\theta}e^{-i\psi}&\cos{\theta} e^{i\phi}
	\end{bmatrix}\\
	=&  \begin{bmatrix}
		\sin{\theta}e^{i\psi} & \cos{\theta} e^{i\phi}\\
		\cos{\theta} e^{-i\phi}&-\sin{\theta}e^{-i\psi}
	\end{bmatrix} \begin{bmatrix}
		\cos{\theta} e^{-i\phi}&-\sin{\theta}e^{i\psi}\\
		\sin{\theta}e^{-i\psi}&\cos{\theta} e^{i\phi}
	\end{bmatrix}\\
	=& \begin{bmatrix}
		\cos\theta \sin\theta\big[e^{i(\psi-\phi)}+e^{i(\phi-\psi)}\big]&\cos^2\theta e^{2i\phi}-\sin^2\theta e^{2i\psi} \\
		\cos^2\theta e^{-2i\phi}-\sin^2\theta e^{-2i\psi}&-\cos\theta \sin\theta\big[e^{i(\psi-\phi)}+e^{i(\phi-\psi)}\big]
	\end{bmatrix} \numberthis \label{eq:U:sigma1:UT}
\end{align*}

The real part of the off diagonal elements of \eqref{eq:U:sigma1:UT} becomes:
\begin{align*}
	&\begin{bmatrix}
		.&\Re(\cos^2\theta e^{2i\phi}-\sin^2\theta e^{2i\psi}) \\
		\Re(\cos^2\theta e^{-2i\phi}-\sin^2\theta e^{-2i\psi})&.
	\end{bmatrix}\\
	=& \frac{1}{2}(\cos^2\theta e^{2i\phi}-\sin^2\theta e^{2i\psi}+ \cos^2\theta e^{-2i\phi}-\sin^2\theta e^{-2i\psi}) \sigma^1	\\
	=& \Re(x^2-y^2)	
\end{align*}

The imaginary part of the off diagonal elements of \eqref{eq:U:sigma1:UT} becomes:
\begin{align*}
	&\begin{bmatrix}
		.&\Im(\cos^2\theta e^{2i\phi}-\sin^2\theta e^{2i\psi}) \\
		\Im(\cos^2\theta e^{-2i\phi}-\sin^2\theta e^{-2i\psi})&.
	\end{bmatrix}\\
	=& \frac{1}{2}(\cos^2\theta e^{2i\phi}-\sin^2\theta e^{2i\psi}- \cos^2\theta e^{-2i\phi}+\sin^2\theta e^{-2i\psi}) \frac{\sigma^2}{i}	\\
	=& -\Im(x^2-y^2)	
\end{align*}

The diagonal elements of \eqref{eq:U:sigma1:UT} are:

\begin{align*}
	 \cos\theta \sin\theta\big[e^{i(\psi-\phi)}+e^{i(\phi-\psi)}\big]\begin{bmatrix}
		1&0\\
		0&-1
	\end{bmatrix}
	=& \cos\theta \sin\theta\big[e^{i(\psi-\phi)}+e^{i(\phi-\psi)}\big] \sigma^3 \text{, now}\\
	\cos\theta \sin\theta e^{i(\psi-\phi)}=& \cos\theta e^{-i\phi} \sin\theta e^{i\psi}\\
	=& \bar{x} y \text{, and}\\
	\cos\theta \sin\theta e^{i(\phi-\psi)}=& \cos\theta e^{i\phi} \sin\theta e^{-i\psi}\\
	=& \bar{x} y \text{, whence} \\
	\cos\theta \sin\theta\big[e^{i(\psi-\phi)}+e^{i(\phi-\psi)}\big] =& x\bar{y} + \bar{x}y\\
	=&2 \Re{x\bar{y}} \text{, and}\\
	\cos\theta \sin\theta\big[e^{i(\psi-\phi)}+e^{i(\phi-\psi)}\big] \sigma^3 =& 2 \Re({x\bar{y}}) \sigma^3 
\end{align*}

So \eqref{eq:U:sigma1:UT} becomes:
\begin{align*}
		U \sigma^1 U^T  =& \Re{(x^2-y^2)} \sigma^1 - \Im{(x^2-y^2)} \sigma^2 + 2 \Re({x\bar{y}}) \sigma^3 
\end{align*}

\begin{align*}
	U \sigma^2 U^T  =&  \begin{bmatrix}
		\cos{\theta} e^{i\phi}&\sin{\theta}e^{i\psi}\\
		-\sin{\theta}e^{-i\psi}&\cos{\theta} e^{-i\phi}
	\end{bmatrix} \begin{bmatrix}
		0&-i\\
		i&0
	\end{bmatrix} \begin{bmatrix}
		\cos{\theta} e^{-i\phi}&-\sin{\theta}e^{i\psi}\\
		\sin{\theta}e^{-i\psi}&\cos{\theta} e^{i\phi}
	\end{bmatrix}
\end{align*}
\begin{align*}
	U \sigma^3 U^T =&  \begin{bmatrix}
		\cos{\theta} e^{i\phi}&\sin{\theta}e^{i\psi}\\
		-\sin{\theta}e^{-i\psi}&\cos{\theta} e^{-i\phi}
	\end{bmatrix} \begin{bmatrix}
		1&0\\
		0&-1
	\end{bmatrix} \begin{bmatrix}
		\cos{\theta} e^{-i\phi}&-\sin{\theta}e^{i\psi}\\
		\sin{\theta}e^{-i\psi}&\cos{\theta} e^{i\phi}
	\end{bmatrix}\\
	=& \begin{bmatrix}
		\cos{\theta} e^{i\phi}&-\sin{\theta}e^{i\psi}\\
		-\sin{\theta}e^{-i\psi}&-\cos{\theta} e^{-i\phi}
	\end{bmatrix}  \begin{bmatrix}
		\cos{\theta} e^{-i\phi}&-\sin{\theta}e^{i\psi}\\
		\sin{\theta}e^{-i\psi}&\cos{\theta} e^{i\phi}
	\end{bmatrix}\\
	=& \begin{bmatrix}
		\cos^2{\theta}-\sin^2{\theta}&-2 \cos{\theta}\sin{\theta}e^{i(\theta+\phi)}\\
		-2 \cos{\theta}\sin{\theta}e^{-i(\theta+\phi)}&-(\cos^2{\theta}-\sin^2{\theta})
	\end{bmatrix} \numberthis \label{eq:U:sigma3:UT}
\end{align*}
Now
\begin{align*}
	xy =& \cos{\theta} e^{i\phi} \sin{\theta} e^{i\psi}\\
	=& \cos{\theta} \sin{\theta}  e^{i(\phi + \psi)}\\
	=& \cos{\theta} \sin{\theta}  \big[\cos{(\phi + \psi)}+ i \sin{(\phi + \psi)}\big]\\
	=& \cos{\theta} \sin{\theta}  \big[\cos{[-(\phi + \psi)]}- i \sin{[-(\phi + \psi)]}\big]
\end{align*}
We see that the real part of the cross term in \eqref{eq:U:sigma3:UT} is:
\begin{align*}
	-2 \cos{\theta}\sin{\theta}\begin{bmatrix}
		.&\Re({e^{i(\phi + \psi)}})\\
		\Re({e^{-i(\phi + \psi)}})&.
	\end{bmatrix}=&	-2 \cos{\theta}\sin{\theta} \begin{bmatrix}
		.&\Re({e^{i(\phi + \psi)}})\\
		\Re({e^{-i(\phi + \psi)}})&.
	\end{bmatrix}\\
	=&	-2 \cos{\theta}\sin{\theta} \begin{bmatrix}
		.&\Re({e^{i(\phi + \psi)}})\\
		\Re({e^{i(\phi + \psi)}})&.\\
	\end{bmatrix} \\
	=& -2 \cos{\theta}\sin{\theta} \Re({e^{i(\phi + \psi)}}) \sigma^1\\
	=& 2 \Re(xy) \sigma^1
\end{align*}

Then the imaginary part of the cross term in \eqref{eq:U:sigma3:UT} is:
\begin{align*}
	-2 \cos{\theta}\sin{\theta}\begin{bmatrix}
		.&\Im({e^{i(\phi + \psi)}})\\
		\Im({e^{-i(\phi + \psi)}})&.
	\end{bmatrix}=&	-2 \cos{\theta}\sin{\theta} \begin{bmatrix}
		.&\Im({e^{i(\phi + \psi)}})\\
		-\Im({e^{i(\phi + \psi)}})&.\\
	\end{bmatrix} \\
	=& 2 \cos{\theta}\sin{\theta} \Im({e^{i(\phi + \psi)}}) \frac{\sigma^2}{i}\\
	=& 2 \Im(xy) \frac{\sigma^2}{i}
\end{align*}
Finally
\begin{align*}
	\vert x \vert^2 - \vert y \vert^2 =& \cos^2 \theta - \sin^2 \theta \text{, so the diagonal term in \eqref{eq:U:sigma3:UT} is}\\
	&(\vert x \vert^2 - \vert y \vert^2) \sigma^3
\end{align*}

So \eqref{eq:U:sigma3:UT} becomes:
\begin{align*}
		U \sigma^3 U^T =&  2 \Re(xy) \sigma^1 + 2 \Im(xy) \sigma^2 + (\vert x \vert^2 - \vert y \vert^2) \sigma^3
\end{align*}

\bibliographystyle{unsrt}
\addcontentsline{toc}{section}{Bibliography}
\raggedright
\bibliography{tm}

\end{document}
