% !TeX program = lualatex

% Copyright (c) 2020-2025 Simon Crase

% Permission is hereby granted, free of charge, to any person obtaining a copy
% of this software and associated documentation files (the "Software"), to deal
% in the Software without restriction, including without limitation the rights
% to use, copy, modify, merge, publish, distribute, sublicense, and/or sell
% copies of the Software, and to permit persons to whom the Software is
% furnished to do so, subject to the following conditions:

% The above copyright notice and this permission notice shall be included in all
% copies or substantial portions of the Software.

% THE SOFTWARE IS PROVIDED "AS IS", WITHOUT WARRANTY OF ANY KIND, EXPRESS OR
% IMPLIED, INCLUDING BUT NOT LIMITED TO THE WARRANTIES OF MERCHANTABILITY,
% FITNESS FOR A PARTICULAR PURPOSE AND NONINFRINGEMENT. IN NO EVENT SHALL THE
% AUTHORS OR COPYRIGHT HOLDERS BE LIABLE FOR ANY CLAIM, DAMAGES OR OTHER
% LIABILITY, WHETHER IN AN ACTION OF CONTRACT, TORT OR OTHERWISE, ARISING FROM,
% OUT OF OR IN CONNECTION WITH THE SOFTWARE OR THE USE OR OTHER DEALINGS IN THE
% SOFTWARE.

\documentclass[]{article}
\usepackage{caption}
\usepackage{subcaption}
\usepackage{graphicx}
\usepackage{float}
\usepackage{url}
\usepackage{amsmath}
\usepackage{amssymb}
\usepackage{amsthm}
\usepackage{tocloft}
\usepackage{cancel}
\usepackage{thmtools}
\usepackage{gensymb}
\usepackage{braket}
\usepackage{bm}
\usepackage[compat=1.1.0]{tikz-feynman}
\usepackage{tikz}
\usepackage{pgfplots}
\usepackage{mathtools}
\usepackage{color}
\usepackage{colortbl}
\usepackage[toc,nonumberlist]{glossaries}
\usepackage{glossaries-extra}
\usepackage{mathtools}

\newcommand\numberthis{\addtocounter{equation}{1}\tag{\theequation}}

\newtheorem{thm}{Theorem}
\newtheorem{defn}[thm]{Definition}
\newtheorem{cor}[thm]{Corollary}
\newtheorem{lemma}[thm]{Lemma}
\graphicspath{{figs/}}
\widowpenalty10000
\clubpenalty10000
\setcounter{tocdepth}{2}
\tikzfeynmanset{compat=1.1.0}
\definecolor{Gray}{gray}{0.5}
\DeclareMathOperator{\Tr}{Tr \;}
\newcommand{\Lagr}{\mathcal{L}}
\renewcommand{\thesection}{2.\arabic{section}}
\setlength{\cftsubsecindent}{0em}
\setlength{\cftsecnumwidth}{3em}
\setlength{\cftsubsecnumwidth}{3em}
%opening
\title{Quantum Field Theory\\
Part II: Dirac and the Spinor}
\author{Simon Crase (compiler)\\simon@greenweaves.nz}

\begin{document}

\maketitle

\begin{abstract}

This document contains derivations of equations from \cite[Part II: Dirac and the Spinor]{zee2010quantum}.

\end{abstract}

\tableofcontents

\section{The Dirac Equation}

\begin{align*}
	(i\gamma^\mu\partial_\mu-m)\psi=&0\\
	\implies&\\
	(i\gamma^\mu\partial_\mu+m)(i\gamma^\nu\partial_\nu-m)\psi=&0 \\
	\iff&\\
	\big(-\gamma^\mu\partial_\mu\gamma^\nu\partial_\nu+\cancel{m i\gamma^\nu\partial_\nu}-\cancel{mi\gamma^\mu\partial_\mu}-m^2\big)\psi=&0\\
	\iff&\\
	\big(\gamma^\mu\partial_\mu\gamma^\nu\partial_\nu+m^2\big)\psi=&0\\
	\iff&\\
	\big(\gamma^\mu\gamma^\nu\partial_\mu\partial_\nu+m^2\big)\psi=&0\\
	\iff&\\
	\big[(\frac{1}{2}(\gamma^\mu\gamma^\nu+\gamma^\nu\gamma^\mu)\partial_\mu\partial_\nu+m^2\big]\psi=&0\\
	\text{Define} \{\gamma^\mu,\gamma^\nu\}=\gamma^\mu\gamma^\nu+\gamma^\nu\gamma^\mu\.\iff&\\
	\big(\frac{1}{2}\{\gamma^\mu,\gamma^\nu\}\partial_\mu\partial_\nu+m^2\big)\psi=&0\\
	\text{setting } \{\gamma^\mu,\gamma^\nu\}=2\eta^{\mu\nu}\.\implies&\\
	(\partial^2 + m^2)\psi =& 0
\end{align*}
\cite{zee2010quantum} defines the $\gamma$s in terms of the Pauli matrices $\sigma/\tau$:
\begin{align*}
	\gamma^0 =& I \otimes \tau_3\\
	\gamma^j =& i \sigma^j \otimes \tau_2
\end{align*}

\subsection{Cousins of the Gamma Matrices}

From \cite[II,(2)]{zee2010quantum}
\begin{align*}
	\{\gamma^\mu,\gamma^\nu\}=&2\eta^{\mu\nu} \numberthis \label{eq:gamma:anti}
\end{align*}
Expanding
\begin{align*} 
	\gamma^\mu\gamma^\nu + \gamma^\nu\gamma^\mu=&2\eta^{\mu\nu}\\
	\gamma^\mu\gamma^\nu =& \eta^{\mu\nu} + \eta^{\mu\nu} -\gamma^\nu\gamma^\mu\\
	=&\eta^{\mu\nu}-i\sigma^{\mu\nu} \text{, where}\\
	\sigma^{\mu\nu}=&i \big(\eta^{\mu\nu} -\gamma^\nu\gamma^\mu\big) \text{. Now apply \eqref{eq:gamma:anti} again}\\
	=&i\big(\frac{1}{2}(\gamma^\mu\gamma^\nu + \gamma^\nu\gamma^\mu)-\gamma^\nu\gamma^\mu\big)\\
	=&i\big(\frac{1}{2}(\gamma^\mu\gamma^\nu - \gamma^\nu\gamma^\mu)\big)\\
	=&\frac{i}{2}[\gamma^\mu,\gamma^\nu]
\end{align*}


\subsection{Isomorphism $\frac{SU(2)}{Z^2} \cong SO(3)$}
In \cite[Section 3]{westra2008SU2} a $2\times 2$ unimodular unitary matrix has the form:
\begin{align*}
	U =& \begin{bmatrix}
		x&-y\\
		\bar{y}&\bar{x}
	\end{bmatrix} \text{. Substituting}\\
	x =& \cos{\theta} e^{i\phi}, \; y = \sin{\theta} e^{i\psi} \numberthis \label{eq:xy} \text{, gives} \\
	U=& \begin{bmatrix}
		\cos{\theta} e^{i\phi}&\sin{\theta}e^{i\psi}\\
		-sin{\theta}e^{-i\psi}&\cos{\theta} e^{-i\phi}
	\end{bmatrix}\\
	U^T =& \begin{bmatrix}
		\cos{\theta} e^{-i\phi}&-\sin{\theta}e^{i\psi}\\
		sin{\theta}e^{-i\psi}&\cos{\theta} e^{i\phi}
	\end{bmatrix}
\end{align*}
We follow \cite[Section 4]{westra2008SU2} and transform the Pauli Matrices.
\begin{align*}
	U \sigma^1 U^T  =&  \begin{bmatrix}
		\cos{\theta} e^{i\phi}&\sin{\theta}e^{i\psi}\\
		-\sin{\theta}e^{-i\psi}&\cos{\theta} e^{-i\phi}
	\end{bmatrix} \begin{bmatrix}
		0&1\\
		1&0
	\end{bmatrix} \begin{bmatrix}
		\cos{\theta} e^{-i\phi}&-\sin{\theta}e^{i\psi}\\
		\sin{\theta}e^{-i\psi}&\cos{\theta} e^{i\phi}
	\end{bmatrix}\\
	=&  \begin{bmatrix}
		\sin{\theta}e^{i\psi} & \cos{\theta} e^{i\phi}\\
		\cos{\theta} e^{-i\phi}&-\sin{\theta}e^{-i\psi}
	\end{bmatrix} \begin{bmatrix}
		\cos{\theta} e^{-i\phi}&-\sin{\theta}e^{i\psi}\\
		\sin{\theta}e^{-i\psi}&\cos{\theta} e^{i\phi}
	\end{bmatrix}\\
	=& \begin{bmatrix}
		\cos\theta \sin\theta\big[e^{i(\psi-\phi)}+e^{i(\phi-\psi)}\big]&\cos^2\theta e^{2i\phi}-\sin^2\theta e^{2i\psi} \\
		\cos^2\theta e^{-2i\phi}-\sin^2\theta e^{-2i\psi}&-\cos\theta \sin\theta\big[e^{i(\psi-\phi)}+e^{i(\phi-\psi)}\big]
	\end{bmatrix} \numberthis \label{eq:U:sigma1:UT}
\end{align*}

The real part of the off diagonal elements of \eqref{eq:U:sigma1:UT} is:
\begin{align*}
	&\begin{bmatrix}
		0&\Re(\cos^2\theta e^{2i\phi}-\sin^2\theta e^{2i\psi}) \\
		\Re(\cos^2\theta e^{-2i\phi}-\sin^2\theta e^{-2i\psi})&0
	\end{bmatrix}\\
	=& \frac{1}{2}(\cos^2\theta e^{2i\phi}-\sin^2\theta e^{2i\psi}+ \cos^2\theta e^{-2i\phi}-\sin^2\theta e^{-2i\psi}) \sigma^1	\\
	=& \Re(x^2-y^2)	
\end{align*}

The imaginary part of the off diagonal elements of \eqref{eq:U:sigma1:UT} is:
\begin{align*}
	&\begin{bmatrix}
		0&\Im(\cos^2\theta e^{2i\phi}-\sin^2\theta e^{2i\psi}) \\
		\Im(\cos^2\theta e^{-2i\phi}-\sin^2\theta e^{-2i\psi})&0
	\end{bmatrix}\\
	=& \frac{1}{2}(\cos^2\theta e^{2i\phi}-\sin^2\theta e^{2i\psi}- \cos^2\theta e^{-2i\phi}+\sin^2\theta e^{-2i\psi}) \frac{\sigma^2}{i}	\\
	=& -\Im(x^2-y^2)	
\end{align*}

The diagonal elements of \eqref{eq:U:sigma1:UT} are:

\begin{align*}
	 \cos\theta \sin\theta\big[e^{i(\psi-\phi)}+e^{i(\phi-\psi)}\big]\begin{bmatrix}
		1&0\\
		0&-1
	\end{bmatrix}
	=& \cos\theta \sin\theta\big[e^{i(\psi-\phi)}+e^{i(\phi-\psi)}\big] \sigma^3 \text{, now}\\
	\cos\theta \sin\theta e^{i(\psi-\phi)}=& \cos\theta e^{-i\phi} \sin\theta e^{i\psi}\\
	=& \bar{x} y \text{, and}\\
	\cos\theta \sin\theta e^{i(\phi-\psi)}=& \cos\theta e^{i\phi} \sin\theta e^{-i\psi}\\
	=& \bar{x} y \text{, whence} \\
	\cos\theta \sin\theta\big[e^{i(\psi-\phi)}+e^{i(\phi-\psi)}\big] =& x\bar{y} + \bar{x}y\\
	=&2 \Re{x\bar{y}} \text{, and}\\
	\cos\theta \sin\theta\big[e^{i(\psi-\phi)}+e^{i(\phi-\psi)}\big] \sigma^3 =& 2 \Re({x\bar{y}}) \sigma^3 
\end{align*}

So \eqref{eq:U:sigma1:UT} becomes:
\begin{align*}
		U \sigma^1 U^T  =& \Re{(x^2-y^2)} \sigma^1 - \Im{(x^2-y^2)} \sigma^2 + 2 \Re({x\bar{y}}) \sigma^3 \numberthis \label{eq:U:sigma1:expanded}
\end{align*}

\begin{align*}
	U \sigma^2 U^T  =&  \begin{bmatrix}
		\cos{\theta} e^{i\phi}&\sin{\theta}e^{i\psi}\\
		-\sin{\theta}e^{-i\psi}&\cos{\theta} e^{-i\phi}
	\end{bmatrix} \begin{bmatrix}
		0&-i\\
		i&0
	\end{bmatrix} \begin{bmatrix}
		\cos{\theta} e^{-i\phi}&-\sin{\theta}e^{i\psi}\\
		\sin{\theta}e^{-i\psi}&\cos{\theta} e^{i\phi}
	\end{bmatrix}\\
	=&  i \begin{bmatrix}
		\cos{\theta} e^{i\phi}&\sin{\theta}e^{i\psi}\\
		-\sin{\theta}e^{-i\psi}&\cos{\theta} e^{-i\phi}
	\end{bmatrix} \begin{bmatrix}
		0&-1\\
		1&0
	\end{bmatrix} \begin{bmatrix}
		\cos{\theta} e^{-i\phi}&-\sin{\theta}e^{i\psi}\\
		\sin{\theta}e^{-i\psi}&\cos{\theta} e^{i\phi}
	\end{bmatrix}\\
	=&  i\begin{bmatrix}
		\sin{\theta}e^{i\psi}&-\cos{\theta} e^{i\phi}\\
		\cos{\theta} e^{-i\phi}&\sin{\theta}e^{-i\psi}
	\end{bmatrix} \begin{bmatrix}
		\cos{\theta} e^{-i\phi}&-\sin{\theta}e^{i\psi}\\
		\sin{\theta}e^{-i\psi}&\cos{\theta} e^{i\phi}
	\end{bmatrix}\\
   =& i \begin{bmatrix}
	   	\sin{\theta}e^{i\psi}\cos{\theta} e^{-i\phi}-\cos{\theta} e^{i\phi}\sin{\theta}e^{-i\psi}&-\sin^2{\theta}e^{2i\psi}-\cos^2{\theta} e^{2i\phi}\\
	   	\cos^2{\theta} e^{-2i\phi}+\sin^2{\theta}e^{-2i\psi}&-\cos{\theta} e^{-i\phi}\sin{\theta}e^{i\psi}+\sin{\theta}e^{-i\psi}\cos{\theta} e^{i\phi}
   \end{bmatrix} \\
	=& i \begin{bmatrix}
		\sin{\theta}\cos{\theta}\big[e^{i(\psi-\phi)}-e^{i(\phi-\psi)}\big]&-\sin^2{\theta}e^{2i\psi}-\cos^2{\theta} e^{2i\phi}\\
		\cos^2{\theta} e^{-2i\phi}+\sin^2{\theta}e^{-2i\psi}&\sin{\theta}\cos{\theta}\big[-e^{i(\psi-\phi)}+e^{i(\phi-\psi)}\big]
	\end{bmatrix} \numberthis \label{eq:U:sigma2:UT}
\end{align*}

The real part of the off diagonal terms in \eqref{eq:U:sigma2:UT} is:
\begin{align*}
	 &\begin{bmatrix}
		0&\Im(\sin^2{\theta}e^{2i\psi}+\cos^2{\theta} e^{2i\phi})\\
		-\Im(\cos^2{\theta} e^{-2i\phi}+\sin^2{\theta}e^{-2i\psi})&0
	\end{bmatrix}\\
   =& \begin{bmatrix}
		0&(\sin^2{\theta}\sin{2i\psi}+\cos^2{\theta} \sin{2i\phi})\\
		-(\cos^2{\theta} \sin({-2i\phi})+\sin^2{\theta}\sin({-2i\psi}))&0
	\end{bmatrix}\\
	=& \begin{bmatrix}
	0&(\sin^2{\theta}\sin{2i\psi}+\cos^2{\theta} \sin{2i\phi})\\
	(\cos^2{\theta} \sin({2i\phi})+\sin^2{\theta}\sin({2i\psi}))&0
	\end{bmatrix}\\
	=& (\sin^2{\theta}\sin{2i\psi}+\cos^2{\theta} \sin{2i\phi}) \sigma^1\\
	=& \Im(x^2 + y^2) \sigma^1
\end{align*}

The imaginary part of the off diagonal terms in \eqref{eq:U:sigma2:UT} is:
\begin{align*}
	\begin{bmatrix}
		0&-\Re({\sin^2{\theta}e^{2i\psi}+\cos^2{\theta} e^{2i\phi}})\\
		\Re({\cos^2{\theta} e^{-2i\phi}+\sin^2{\theta}e^{-2i\psi}})&0
	\end{bmatrix}
	=&\begin{bmatrix}
		0&-\Re(x^2+y^2)\\
		\Re(\bar{x}^2+\bar{y}^2)&0
	\end{bmatrix}\\
	=& \frac{\Re(x^2+y^2)}{i} \sigma_2
\end{align*}

The diagonal terms in \eqref{eq:U:sigma2:UT} are:
\begin{align*}
	i \sin{\theta}\cos{\theta}\big[e^{i(\psi-\phi)}-e^{i(\phi-\psi)}\big] \sigma^3 =& i \big[\bar{x}y-x\bar{y}\big] \sigma^3\\
	=&i 2 i  \Im({\bar{x}y}) \sigma^3\\
	=& - 2 \Im({\bar{x}y}) \sigma^3
\end{align*}

So \eqref{eq:U:sigma2:UT} becomes:
\begin{align*}
	U \sigma^2 U^T  =& \Im(x^2 + y^2) \sigma^1 + \Re(x^2+y^2) \sigma_2 - 2 \Im({\bar{x}y}) \sigma^3  \numberthis \label{eq:U:sigma2:expanded}
\end{align*}

\begin{align*}
	U \sigma^3 U^T =&  \begin{bmatrix}
		\cos{\theta} e^{i\phi}&\sin{\theta}e^{i\psi}\\
		-\sin{\theta}e^{-i\psi}&\cos{\theta} e^{-i\phi}
	\end{bmatrix} \begin{bmatrix}
		1&0\\
		0&-1
	\end{bmatrix} \begin{bmatrix}
		\cos{\theta} e^{-i\phi}&-\sin{\theta}e^{i\psi}\\
		\sin{\theta}e^{-i\psi}&\cos{\theta} e^{i\phi}
	\end{bmatrix}\\
	=& \begin{bmatrix}
		\cos{\theta} e^{i\phi}&-\sin{\theta}e^{i\psi}\\
		-\sin{\theta}e^{-i\psi}&-\cos{\theta} e^{-i\phi}
	\end{bmatrix}  \begin{bmatrix}
		\cos{\theta} e^{-i\phi}&-\sin{\theta}e^{i\psi}\\
		\sin{\theta}e^{-i\psi}&\cos{\theta} e^{i\phi}
	\end{bmatrix}\\
	=& \begin{bmatrix}
		\cos^2{\theta}-\sin^2{\theta}&-2 \cos{\theta}\sin{\theta}e^{i(\theta+\phi)}\\
		-2 \cos{\theta}\sin{\theta}e^{-i(\theta+\phi)}&-(\cos^2{\theta}-\sin^2{\theta})
	\end{bmatrix} \numberthis \label{eq:U:sigma3:UT}
\end{align*}
Now
\begin{align*}
	xy =& \cos{\theta} e^{i\phi} \sin{\theta} e^{i\psi}\\
	=& \cos{\theta} \sin{\theta}  e^{i(\phi + \psi)}\\
	=& \cos{\theta} \sin{\theta}  \big[\cos{(\phi + \psi)}+ i \sin{(\phi + \psi)}\big]\\
	=& \cos{\theta} \sin{\theta}  \big[\cos{[-(\phi + \psi)]}- i \sin{[-(\phi + \psi)]}\big]
\end{align*}
We see that the real part of the cross term in \eqref{eq:U:sigma3:UT} is:
\begin{align*}
	-2 \cos{\theta}\sin{\theta}\begin{bmatrix}
		0&\Re({e^{i(\phi + \psi)}})\\
		\Re({e^{-i(\phi + \psi)}})&0
	\end{bmatrix}=&	-2 \cos{\theta}\sin{\theta} \begin{bmatrix}
		0&\Re({e^{i(\phi + \psi)}})\\
		\Re({e^{-i(\phi + \psi)}})&0
	\end{bmatrix}\\
	=&	-2 \cos{\theta}\sin{\theta} \begin{bmatrix}
		.&\Re({e^{i(\phi + \psi)}})\\
		\Re({e^{i(\phi + \psi)}})&.\\
	\end{bmatrix} \\
	=& -2 \cos{\theta}\sin{\theta} \Re({e^{i(\phi + \psi)}}) \sigma^1\\
	=& 2 \Re(xy) \sigma^1
\end{align*}

Then the imaginary part of the cross term in \eqref{eq:U:sigma3:UT} is:
\begin{align*}
	-2 \cos{\theta}\sin{\theta}\begin{bmatrix}
		0&\Im({e^{i(\phi + \psi)}})\\
		\Im({e^{-i(\phi + \psi)}})&0
	\end{bmatrix}=&	-2 \cos{\theta}\sin{\theta} \begin{bmatrix}
		0&\Im({e^{i(\phi + \psi)}})\\
		-\Im({e^{i(\phi + \psi)}})&0\\
	\end{bmatrix} \\
	=& 2 \cos{\theta}\sin{\theta} \Im({e^{i(\phi + \psi)}}) \frac{\sigma^2}{i}\\
	=& 2 \Im(xy) \frac{\sigma^2}{i}
\end{align*}
Finally
\begin{align*}
	\vert x \vert^2 - \vert y \vert^2 =& \cos^2 \theta - \sin^2 \theta \text{, so the diagonal term in \eqref{eq:U:sigma3:UT} is}\\
	&(\vert x \vert^2 - \vert y \vert^2) \sigma^3
\end{align*}

So \eqref{eq:U:sigma3:UT} becomes:
\begin{align*}
		U \sigma^3 U^T =&  2 \Re(xy) \sigma^1 + 2 \Im(xy) \sigma^2 + (\vert x \vert^2 - \vert y \vert^2) \sigma^3 \numberthis \label{eq:U:sigma3:expanded}
\end{align*}

Collecting \eqref{eq:U:sigma1:expanded}, \eqref{eq:U:sigma2:expanded}, and \eqref{eq:U:sigma3:expanded}, we get:
\begin{equation}
	\begin{aligned}
		U \sigma^1 U^T  =& \Re{(x^2-y^2)} \sigma^1 -& \Im{(x^2-y^2)} \sigma^2 +& 2 \Re({x\bar{y}}) \sigma^3\\
		U \sigma^2 U^T  =& \Im(x^2 + y^2) \sigma^1 +& \Re(x^2+y^2) \sigma_2 -& 2 \Im({\bar{x}y}) \sigma^3  \\
		U \sigma^3 U^T =&  2 \Re(xy) \sigma^1 +& 2 \Im(xy) \sigma^2 +& (\vert x \vert^2 - \vert y \vert^2) \sigma^3
	\end{aligned}\label{eq:all3sigma}
\end{equation}

\section{Quantizing the Dirac Field}
\section{Lorentz Group and Weyl Spinors}
\section{Spin-Statistics Connection}
\section{Vacuum Energy, etc}
\section{Electron Scattering and Gauge Invariance}
\section{Diagrammatic Proof of Gauge Invariance}
\section{Photon-Electron Scattering}

\bibliographystyle{unsrt}
\addcontentsline{toc}{section}{Bibliography}
\raggedright
\bibliography{tm}

\end{document}
