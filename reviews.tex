
\documentclass[]{article}

\usepackage{url}
%opening
\title{Helgoland}
\author{Simon Crase (reviewer)\\simon@greenweaves.nz}

\begin{document}
	
\maketitle

Helgoland is a treeless, wind battered island in the North Sea. In 1925 Werner Heisenberg retreated there, hoping for a respite from hay fever so he could invent quantum mechanics. \emph{Helgoland} is also the title of Carlo Rovelli's newest book\cite{rovelli2020helgoland}.

The first part deals with the history surrounding Heisenberg's historic paper\cite{heisenberg1925quantum}. Rutherford had bombarded atoms with alpha patticles, and explained the results with a model with electrons orbiting the nucleus. Bohr took the model further to explain the frequencies observed in the spectrum of hydrogen: he assumed, for no good reason, that electrons moved in certain precise orbits, at certain precise distances from the nucleus, with certain precise energies, before magically leaping from that orbit to another, thereby emitting light of just the right frequency. Heisenberg was brought in to explain the orbits: instead he threw them out. On Helgoland he had an epiphany:  scrap the orbits altogether, and describe atomic systems by observables only; he threw out everything unobservable, and stuck with what could be observed and measured. He found he could make this work by replacing the numbers that encoded observations with \emph{tables} of numbers, that combined in accordance with rules that he devised.

Max Born quickly realized that he had seen Heisenberg's tables before: it was "matrix multiplication", which was an esoteric mathematical technique 100 years ago. The following year Scr\"odinger came up with his wave mechanics in 1926, which used differential equations, the standard tool of theoretical physics at the time\footnote{When I did my degree, there was an honours paper entitled "Methods of Applied Mathematics"; it was renamed in 1969 to "Differential Equations"--as that was what it really was.}. Rovelli regrets the distortions of quantum mechanics: the idea that electrons are waves, that the wave function somehow "collapses" when we make a measurement (how does the electron \emph{know}?); the box with its bewildered cat.

The next part of \emph{Helgoland} outlines Rovelli's interpretation of quantum mechanics, Relational Quantum Mechanics\cite{rovelli2018space}\cite{sep-qm-relational}: properties of a object have no meaning, except in their interaction with something else. Galileo realized this for velocities: my velocity only has meaning if I measure it relative to the road, or to another car, or something else. Scr\"odinger's cat may be alive or asleep\footnote{Rovelli is a cat lover, and doesn't like to kill one even in a thought experiment} inside its box, but it's a different story outside.

Rovelli is very well read: he knows his Cicero, Dante, and Lucretius, and has previously written of Anaximander as the first scientist \cite{rovelli2011anaximander}. In \emph{Helgoland} he discusses the influence of Mach on both Einstein and Heisenberg. The least satisfactory part of \emph{Helgoland} is the digression on Aleksander Aleksandrovich Bogdanov, another disciple of Mach, who ran afoul of Lenin and Stalin. Rovelli also discusses Nagarjuna, a Buddhist teacher from the 2nd century CE, who taught that nothing exists for itself, independently from everything else. \begin{quotation}
	What really interests us about ancient texts is not what the author initially intended to say: it is how the work can speak to us now, and what it can suggest today.
\end{quotation}  

\bibliographystyle{unsrt}
\bibliography{reviews}

\end{document}